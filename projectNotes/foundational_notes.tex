\documentclass[12pt,a4paper]{article}

% ================== Packages ==================
\usepackage{amsmath,amssymb}
\usepackage{geometry}
\usepackage{setspace}
\usepackage{hyperref}

\geometry{margin=1in}
\onehalfspacing

\begin{document}

% ================== Title ==================
\begin{center}
    {\Large \textbf{Foundation Notes}}\\[0.5em]
    {\large Preparing for Waves, Vibrations, and Rayleigh Theory}\\[1em]
    {\Large \textbf{Chapter 0: What Physics Is Actually Doing}}
\end{center}

\vspace{1.5em}

% ================== Introduction ==================
\section*{Why this chapter exists}

Before we learn equations, symbols, or formulas, we must answer one very important question:

\begin{quote}
\textbf{What is physics trying to do in the first place?}
\end{quote}

Most students struggle with physics not because the mathematics is too hard,  
but because they never clearly understood \emph{why} mathematics appears at all.

This chapter exists to remove that confusion completely.

No formulas will be used yet.  
No calculations will be required.

We are only learning how to \emph{think}.



% ================== Section ==================
\section{Reality vs Description}

The real world contains:
\begin{itemize}
    \item moving objects,
    \item vibrating ground,
    \item flowing air,
    \item changing temperatures.
\end{itemize}

Physics does \emph{not} contain these things.

Physics contains \textbf{descriptions} of these things.

This distinction is crucial.

\begin{quote}
\textbf{Reality is not an equation.  
An equation is a description of reality.}
\end{quote}



% ================== Models ==================
\section{What Is a Model?}

A \textbf{model} is a simplified description of something real.

Examples:
\begin{itemize}
    \item A map is a model of a city.
    \item A globe is a model of the Earth.
    \item A graph is a model of motion.
\end{itemize}

A model is never perfect.

It is useful if it captures what matters and ignores what does not.

Physics is the art of choosing good models.



% ================== Why Math ==================
\section{Why Physics Uses Mathematics}

Words are good for stories.

Mathematics is good for:
\begin{itemize}
    \item precision,
    \item prediction,
    \item compression of ideas.
\end{itemize}

The sentence:
\begin{quote}
``The motion changes smoothly in time''
\end{quote}

can be written compactly as:
\begin{quote}
``a function of time''
\end{quote}

Mathematics is not used to confuse.  
It is used to say more with fewer words.



% ================== Symbols ==================
\section{What Symbols Really Are}

A symbol is simply a \emph{name}.

For example:
\begin{itemize}
    \item $t$ is a name for time,
    \item $x$ is a name for position,
    \item $u$ might be a name for motion.
\end{itemize}

Nothing mystical is happening.

Writing:
\[
x(t)
\]
just means:
\begin{quote}
``The position depends on time.''
\end{quote}

That is all.



% ================== Functions ==================
\section{What a Function Means}

A function answers the question:

\begin{quote}
\textbf{If I tell you one thing, can you tell me another?}
\end{quote}

Examples:
\begin{itemize}
    \item If I tell you the time, can you tell me the position?
    \item If I tell you the day, can you tell me the temperature?
\end{itemize}

A function is not a formula first.  
It is a \emph{relationship} first.

Formulas come later.



% ================== Change ==================
\section{Why Change Matters}

Physics is interested in change.

Things that never change are boring to physics.

Examples of change:
\begin{itemize}
    \item motion,
    \item vibration,
    \item sound,
    \item waves.
\end{itemize}

Later, we will measure change very precisely.  
For now, it is enough to understand that physics studies how things vary.



% ================== Prediction ==================
\section{What Physics Tries to Predict}

A good physical model allows us to answer questions like:
\begin{itemize}
    \item Where will this object be later?
    \item How fast will it move?
    \item How strong will the vibration be?
\end{itemize}

Physics is not about memorizing formulas.

Physics is about:
\begin{quote}
\textbf{understanding patterns well enough to predict them.}
\end{quote}



% ================== Fear ==================
\section{About Fear of Mathematics}

Many students fear symbols.

This fear usually comes from seeing equations \emph{before} understanding meaning.

In these notes:
\begin{itemize}
    \item meaning always comes first,
    \item symbols are introduced slowly,
    \item every equation earns its place.
\end{itemize}

You are not expected to accept anything blindly.



% ================== Looking Ahead ==================
\section*{Looking Ahead}

In the next chapter, we will begin with the simplest kind of change:

\begin{quote}
\textbf{Motion that goes back and forth.}
\end{quote}

This kind of motion is called \emph{oscillation}.

Oscillation is the heart of:
\begin{itemize}
    \item sound,
    \item waves,
    \item earthquakes,
    \item Rayleigh waves.
\end{itemize}

% ================== Title ==================
\pagebreak
\begin{center}
    {\Large \textbf{Chapter 1: Motion and Change}}
\end{center}

\vspace{1.5em}

% ================== Introduction ==================
\section*{Why this chapter exists}

Physics is the study of \textbf{change}.

Before we talk about waves, vibrations, or earthquakes,  
we must understand the simplest kind of change:

\begin{quote}
\textbf{How position changes with time.}
\end{quote}

This chapter introduces motion in a way that feels natural, visual, and intuitive.



% ================== Motion ==================
\section{What Is Motion?}

An object is said to be in motion if its position changes with time.

Examples:
\begin{itemize}
    \item a car moving on a road,
    \item a person walking,
    \item a point on a vibrating string.
\end{itemize}

To describe motion, we must answer two questions:
\begin{itemize}
    \item Where is the object?
    \item When is it there?
\end{itemize}



% ================== Position ==================
\section{Position as a Number}

To describe position, we choose a reference line.

Along this line:
\begin{itemize}
    \item each location is assigned a number,
    \item positive and negative directions are defined.
\end{itemize}

We call this number the \textbf{position} and usually denote it by:
\[
x
\]



% ================== Time ==================
\section{Time as a Variable}

Time tells us \emph{when} something happens.

We denote time by:
\[
t
\]

Time flows forward, and we measure it using clocks.



% ================== Position vs Time ==================
\section{Position Depends on Time}

In most situations, position is not fixed.

It changes as time passes.

We express this idea by writing:
\[
x(t)
\]

This is read as:
\begin{quote}
``Position as a function of time.''
\end{quote}

It does \emph{not} mean multiplication.

It means dependence.



% ================== Graphs ==================
\section{Graphs Tell Stories}

One of the most powerful tools in physics is a graph.

A graph of position versus time shows:
\begin{itemize}
    \item time on the horizontal axis,
    \item position on the vertical axis.
\end{itemize}

Each point on the graph answers the question:
\begin{quote}
\textbf{Where was the object at this moment in time?}
\end{quote}



% ================== Straight Line Motion ==================
\section{Constant Motion}

If an object moves equal distances in equal times,
its position–time graph is a straight line.

This means:
\begin{itemize}
    \item the motion is steady,
    \item the change is uniform.
\end{itemize}

The steepness of the line tells us how fast the object is moving.



% ================== Speed ==================
\section{Speed as Rate of Change}

Speed measures how quickly position changes.

On a position–time graph:
\begin{quote}
\textbf{Speed is the slope of the graph.}
\end{quote}

Steeper slope means higher speed.

Gentler slope means slower motion.



% ================== Velocity ==================
\section{Direction Matters}

Speed tells us how fast.

Velocity tells us:
\begin{itemize}
    \item how fast,
    \item in which direction.
\end{itemize}

A positive slope means motion in the positive direction.

A negative slope means motion in the opposite direction.



% ================== Changing Speed ==================
\section{When Motion Is Not Constant}

If the graph is curved:
\begin{itemize}
    \item speed is changing,
    \item motion is not uniform.
\end{itemize}

This happens when:
\begin{itemize}
    \item an object accelerates,
    \item an object slows down,
    \item motion reverses direction.
\end{itemize}



% ================== Idea of Derivative ==================
\section{The Idea of Instantaneous Change}

Sometimes we want to know:
\begin{quote}
\textbf{How fast is the object moving at one exact moment?}
\end{quote}

This is called \textbf{instantaneous velocity}.

We find it by looking at:
\begin{itemize}
    \item the slope of the graph at one point,
    \item not over a long interval.
\end{itemize}

Later, this idea becomes the derivative.

For now, it is enough to understand it visually.



% ================== Why This Matters ==================
\section{Why Motion Matters for Waves}

Waves are made of many points moving back and forth.

If we cannot describe the motion of one point,
we cannot describe a wave.

Everything that follows — oscillations, waves, energy —  
is built on this foundation.



% ================== Summary ==================
\section*{What You Should Know Now}

After this chapter, you should be comfortable with:
\begin{itemize}
    \item position as a number,
    \item time as a variable,
    \item position as a function of time,
    \item graphs as descriptions of motion,
    \item speed as rate of change.
\end{itemize}

No formulas were memorized.

Everything was understood.



% ================== Looking Ahead ==================
\section*{Looking Ahead}

In the next chapter, we will study a very special kind of motion:

\begin{quote}
\textbf{Motion that repeats itself.}
\end{quote}

This is called \emph{oscillation},  
and it is the heart of waves and vibrations.
\pagebreak
% ================== Title ==================
\begin{center}
    {\Large \textbf{Chapter 2: Repetition and Oscillation}}
\end{center}

\vspace{1.5em}

% ================== Introduction ==================
\section*{Why this chapter exists}

In the previous chapter, we studied motion that moves forward.

Now we study a very special kind of motion:

\begin{quote}
\textbf{Motion that repeats itself.}
\end{quote}

This kind of motion is called \emph{oscillation}.

Oscillation is the foundation of:
\begin{itemize}
    \item sound,
    \item musical instruments,
    \item waves,
    \item earthquakes,
    \item Rayleigh waves.
\end{itemize}

If oscillation is understood well, everything later becomes easier.



% ================== Repetition ==================
\section{What Does It Mean to Repeat?}

Something repeats if it goes through the same sequence of motion again and again.

Examples:
\begin{itemize}
    \item a pendulum swinging,
    \item a mass on a spring,
    \item a child on a swing,
    \item a vibrating guitar string.
\end{itemize}

After some time, the motion looks exactly the same as before.



% ================== Cycle ==================
\section{One Complete Cycle}

A \textbf{cycle} is one complete repetition of motion.

For a swinging pendulum:
\begin{itemize}
    \item starting from the center,
    \item going to one side,
    \item coming back to the center,
    \item going to the other side,
    \item returning to the center.
\end{itemize}

That entire motion is one cycle.



% ================== Period ==================
\section{Period: How Long One Cycle Takes}

The \textbf{period} is the time taken to complete one cycle.

We usually denote the period by:
\[
T
\]

If one cycle takes 2 seconds, then:
\[
T = 2 \text{ seconds}
\]



% ================== Frequency ==================
\section{Frequency: How Fast Repetition Happens}

Instead of asking:
\begin{quote}
``How long does one cycle take?''
\end{quote}

we can ask:
\begin{quote}
``How many cycles happen in one second?''
\end{quote}

This is called the \textbf{frequency}.

Frequency is denoted by:
\[
f
\]



% ================== Relation ==================
\section{Relation Between Period and Frequency}

Period and frequency describe the same thing in two different ways.

If one cycle takes $T$ seconds, then the number of cycles per second is:
\[
f = \frac{1}{T}
\]

If the period is large:
\begin{itemize}
    \item repetition is slow,
    \item frequency is small.
\end{itemize}

If the period is small:
\begin{itemize}
    \item repetition is fast,
    \item frequency is large.
\end{itemize}



% ================== Examples ==================
\section{Examples from Everyday Life}

\subsection*{Clocks}
A clock ticking once per second has:
\[
T = 1 \text{ s}, \quad f = 1 \text{ Hz}
\]

\subsection*{Music}
A musical note with frequency 440 Hz means:
\begin{itemize}
    \item 440 oscillations every second.
\end{itemize}

Higher pitch means higher frequency.



% ================== Why Physics Loves Oscillations ==================
\section{Why Oscillations Are Special in Physics}

Oscillations are important because:
\begin{itemize}
    \item they are predictable,
    \item they are stable,
    \item they can store and transfer energy.
\end{itemize}

Many complicated motions can be broken into simple oscillations.

This idea becomes very powerful later.



% ================== Oscillation in Space ==================
\section{Oscillation Is Motion About a Balance Point}

In oscillatory motion:
\begin{itemize}
    \item the object moves back and forth,
    \item there is a central position,
    \item motion happens around that position.
\end{itemize}

This central position is called the \textbf{equilibrium}.



% ================== Small Oscillations ==================
\section{Small Oscillations}

If oscillations are small:
\begin{itemize}
    \item the motion is smooth,
    \item the pattern repeats cleanly.
\end{itemize}

This is why physics often studies small oscillations first.

They reveal the essential behavior.



% ================== Why This Matters ==================
\section{Why Oscillation Matters for Waves}

A wave is made of many points oscillating:
\begin{itemize}
    \item each point repeats motion in time,
    \item neighboring points repeat with a delay.
\end{itemize}

Understanding one oscillation prepares us to understand waves.



% ================== Summary ==================
\section*{What You Should Know Now}

After this chapter, you should be comfortable with:
\begin{itemize}
    \item what oscillation means,
    \item what a cycle is,
    \item the meaning of period $T$,
    \item the meaning of frequency $f$,
    \item the relation $f = 1/T$.
\end{itemize}

These ideas will appear everywhere later.



% ================== Looking Ahead ==================
\section*{Looking Ahead}

In the next chapter, we will look at a very special shape:

\begin{quote}
\textbf{The sine wave.}
\end{quote}

This shape describes oscillation mathematically and leads directly to wave theory.
\pagebreak
% ================== Title ==================
\begin{center}
    {\Large \textbf{Chapter 3: Sine and Cosine as Shapes}}
\end{center}

\vspace{1.5em}

% ================== Introduction ==================
\section*{Why this chapter exists}

In the previous chapter, we learned that oscillation is motion that repeats.

Now we ask a natural question:

\begin{quote}
\textbf{What does repeating motion look like when we draw it?}
\end{quote}

The answer leads us to two very special shapes:
\begin{itemize}
    \item the sine curve,
    \item the cosine curve.
\end{itemize}

These shapes appear everywhere in wave physics.



% ================== Drawing Motion ==================
\section{Drawing Oscillatory Motion}

Imagine a point that moves:
\begin{itemize}
    \item up,
    \item down,
    \item up,
    \item down,
\end{itemize}
in a smooth and regular way.

If we draw its position versus time, we get a smooth, repeating curve.

This curve is called a \textbf{sine wave}.



% ================== Shape First ==================
\section{Sine as a Shape, Not a Formula}

At this stage, forget formulas.

Think only about shape.

A sine curve:
\begin{itemize}
    \item repeats itself,
    \item is smooth,
    \item has no sharp corners,
    \item goes equally high and low.
\end{itemize}

This makes it perfect for describing smooth oscillations.



% ================== Amplitude ==================
\section{Amplitude: How Big the Motion Is}

The maximum distance from the center position is called the \textbf{amplitude}.

We usually denote amplitude by:
\[
A
\]

A larger amplitude means:
\begin{itemize}
    \item bigger motion,
    \item more energy.
\end{itemize}

A smaller amplitude means gentler motion.



% ================== Center ==================
\section{The Middle Position}

Sine and cosine oscillate around a central value.

Often this central value is zero.

This corresponds to oscillation about equilibrium.



% ================== Period Again ==================
\section{Period Revisited}

The sine curve repeats after a fixed time.

This time is exactly the \textbf{period} $T$.

The shape looks the same after every interval of length $T$.



% ================== Phase ==================
\section{Phase: Where You Are in the Cycle}

Two sine waves can have the same shape but start differently.

This difference is called the \textbf{phase}.

Phase tells us:
\begin{itemize}
    \item where the motion is in its cycle,
    \item whether it starts at the center or the peak.
\end{itemize}

Phase will become very important later.



% ================== Sine vs Cosine ==================
\section{Sine and Cosine Together}

Sine and cosine have the same shape.

The only difference:
\begin{itemize}
    \item sine starts at zero,
    \item cosine starts at a maximum.
\end{itemize}

They are just shifted versions of each other.



% ================== Writing the Shape ==================
\section{Writing the Shape Mathematically}

Now we allow ourselves to write:
\[
x(t) = A \sin(\text{something})
\]

This means:
\begin{quote}
``Position varies in a smooth, repeating way.''
\end{quote}

The exact contents of ``something'' will be explained soon.



% ================== Why Sine ==================
\section{Why Sine Waves Appear Everywhere}

Sine waves appear because:
\begin{itemize}
    \item they repeat smoothly,
    \item their shape does not change over time,
    \item they combine well with each other.
\end{itemize}

Nature prefers smoothness.

Sine waves provide that.



% ================== Superposition ==================
\section{Combining Oscillations (Preview)}

If two sine waves are added:
\begin{itemize}
    \item the result is still smooth,
    \item patterns remain predictable.
\end{itemize}

Later, we will see that complicated motion can be built from simple sine waves.



% ================== Why This Matters ==================
\section{Why This Matters for Waves}

Every wave we study later:
\begin{itemize}
    \item uses sine or cosine in time,
    \item uses sine or cosine in space,
    \item combines both smoothly.
\end{itemize}

This chapter gives the shape that waves are built from.



% ================== Summary ==================
\section*{What You Should Know Now}

After this chapter, you should:
\begin{itemize}
    \item recognize a sine wave by shape,
    \item understand amplitude,
    \item understand phase qualitatively,
    \item see sine and cosine as shifted versions.
\end{itemize}

No trigonometry tricks were required.

Only understanding.



% ================== Looking Ahead ==================
\section*{Looking Ahead}

In the next chapter, we will ask:

\begin{quote}
\textbf{How fast does a sine wave oscillate?}
\end{quote}

This will lead us to angular frequency and time-harmonic motion.

\pagebreak

% ================== Title ==================
\begin{center}
    {\Large \textbf{Chapter 4: Time-Harmonic Motion}}
\end{center}

\vspace{1.5em}

% ================== Motivation ==================
\section*{Why this chapter exists}

So far, we have learned:
\begin{itemize}
    \item how motion is described,
    \item what oscillation means,
    \item how sine and cosine describe repeating motion.
\end{itemize}

Now we take a crucial step.

We decide to focus on a very special kind of motion:

\begin{quote}
\textbf{Motion that repeats forever with a single frequency.}
\end{quote}

This is called \textbf{time-harmonic motion}.

This choice is not arbitrary.
It is the foundation of all wave theory.



% ================== Single Frequency ==================
\section{What Does “Single Frequency” Mean?}

Consider an oscillation that:
\begin{itemize}
    \item repeats perfectly,
    \item never changes its shape,
    \item never changes its rhythm.
\end{itemize}

Such motion has exactly \emph{one} frequency.

This means:
\begin{itemize}
    \item every cycle takes the same amount of time,
    \item the motion is perfectly regular.
\end{itemize}

This is the cleanest motion nature allows.



% ================== Mathematical Expression ==================
\section{Writing Time-Harmonic Motion}

We write time-harmonic motion as:
\[
x(t) = A \sin(\omega t)
\]

Let us interpret every symbol carefully.

\begin{itemize}
    \item $x(t)$ is the position at time $t$
    \item $A$ is the amplitude (how large the motion is)
    \item $\omega$ controls how fast the oscillation happens
\end{itemize}

Nothing in this equation is mysterious.



% ================== Angular Frequency ==================
\section{Why We Use Angular Frequency}

Earlier, we defined frequency as:
\[
f = \frac{1}{T}
\]

Physics almost always uses:
\[
\omega = 2\pi f
\]

Why?

Because sine and cosine naturally repeat every $2\pi$.

Using $\omega$:
\begin{itemize}
    \item simplifies equations,
    \item removes unnecessary constants,
    \item makes calculus cleaner.
\end{itemize}

This is a mathematical convenience, not a new physical idea.



% ================== Units ==================
\section{Units and Meaning}

Frequency $f$ is measured in:
\[
\text{cycles per second (Hz)}
\]

Angular frequency $\omega$ is measured in:
\[
\text{radians per second}
\]

Both describe how fast oscillation occurs.
They carry the same information.



% ================== Phase ==================
\section{Phase: Shifting the Oscillation}

A more general harmonic motion is:
\[
x(t) = A \sin(\omega t + \phi)
\]

Here $\phi$ is called the \textbf{phase}.

Phase does not change:
\begin{itemize}
    \item the frequency,
    \item the amplitude,
    \item the shape.
\end{itemize}

It only changes \emph{where} the motion starts in time.



% ================== Graphical Meaning ==================
\section{Graphical Interpretation}

On a graph of position versus time:
\begin{itemize}
    \item amplitude controls height,
    \item frequency controls spacing of peaks,
    \item phase controls horizontal shift.
\end{itemize}

Every parameter has a clear geometric meaning.



% ================== Why Physics Loves Harmonic Motion ==================
\section{Why Harmonic Motion Is Central in Physics}

Physics focuses on time-harmonic motion because:

\begin{itemize}
    \item many systems naturally oscillate this way,
    \item complicated motion can be decomposed into harmonics,
    \item equations become solvable.
\end{itemize}

This idea will later become:
\begin{quote}
\textbf{Fourier analysis.}
\end{quote}



% ================== Preview of Differentiation ==================
\section{A First Look at Differentiation}

If:
\[
x(t) = \sin(\omega t)
\]

then:
\[
\frac{dx}{dt} \propto \cos(\omega t)
\]

and:
\[
\frac{d^2x}{dt^2} \propto -\sin(\omega t)
\]

The shape stays the same.

This property will become extremely important.



% ================== Why This Matters for Waves ==================
\section{Why Time-Harmonic Motion Leads to Waves}

A wave is built from:
\begin{itemize}
    \item oscillation in time,
    \item variation in space.
\end{itemize}

If we understand time-harmonic motion,
we only need to add \emph{space dependence} to obtain waves.

That is exactly what we will do later.



% ================== Summary ==================
\section*{What You Should Know Now}

After this chapter, you should clearly understand:
\begin{itemize}
    \item what time-harmonic motion is,
    \item why $\omega$ is used instead of $f$,
    \item what amplitude and phase mean,
    \item why harmonic motion is central to physics.
\end{itemize}

Nothing here was arbitrary.



% ================== Looking Ahead ==================
\section*{Looking Ahead}

In the next chapter, we will face an honest problem:

\begin{quote}
\textbf{Real sine and cosine functions make calculations messy.}
\end{quote}

This problem will force us to introduce a powerful new idea.
\pagebreak
% ================== Title ==================
\begin{center}
    {\Large \textbf{Chapter 5: Why Real Sines Are Inconvenient}}
\end{center}

\vspace{1.5em}

% ================== Introduction ==================
\section*{Why this chapter exists}

Up to now, we have described oscillations using sine and cosine.

This works well — but not perfectly.

In this chapter, we will discover a problem:

\begin{quote}
\textbf{Real sine and cosine functions are conceptually simple,  
but mathematically inconvenient.}
\end{quote}

This problem will naturally lead us to a better tool.



% ================== Differentiation ==================
\section{What Happens When We Differentiate a Sine?}

Consider the time-harmonic motion:
\[
x(t) = A \sin(\omega t)
\]

Its rate of change is:
\[
\frac{dx}{dt} = A \omega \cos(\omega t)
\]

Differentiating again:
\[
\frac{d^2x}{dt^2} = -A \omega^2 \sin(\omega t)
\]

Notice what happened:
\begin{itemize}
    \item sine became cosine,
    \item cosine became sine again,
    \item signs changed.
\end{itemize}

The shape stays the same, but the expression keeps changing.



% ================== Algebraic Mess ==================
\section{Why This Becomes Messy}

In wave physics, we often:
\begin{itemize}
    \item differentiate many times,
    \item add oscillations,
    \item multiply by constants,
    \item solve differential equations.
\end{itemize}

Using sine and cosine separately causes:
\begin{itemize}
    \item extra terms,
    \item phase confusion,
    \item longer calculations.
\end{itemize}

The physics stays simple.
The algebra does not.



% ================== Phase Shifts ==================
\section{Phase Shifts Are Awkward with Sines}

Consider:
\[
x(t) = A \sin(\omega t + \phi)
\]

Expanding this requires identities:
\[
\sin(\omega t + \phi) =
\sin(\omega t)\cos\phi + \cos(\omega t)\sin\phi
\]

Now one oscillation becomes two.

This is mathematically correct, but cumbersome.



% ================== Combining Oscillations ==================
\section{Adding Oscillations}

Suppose we add two oscillations:
\[
x(t) = A_1 \sin(\omega t) + A_2 \cos(\omega t)
\]

This represents a single oscillation with a phase shift.

But to see that requires more algebra.

The physical idea is simple.
The mathematics is not.



% ================== A Wish ==================
\section{What We Wish For}

Ideally, we want a mathematical tool where:
\begin{itemize}
    \item differentiation does not change the shape,
    \item phase shifts are easy,
    \item algebra stays clean.
\end{itemize}

In other words, we want:
\begin{quote}
\textbf{a function that reproduces itself under differentiation.}
\end{quote}



% ================== Preview ==================
\section{A Hint of What Is Coming}

There exists a function with the magical property:
\[
\frac{d}{dt}(\text{function}) = \text{constant} \times (\text{same function})
\]

This function will allow us to:
\begin{itemize}
    \item replace trigonometry with algebra,
    \item simplify wave equations,
    \item think more clearly.
\end{itemize}

We will meet it in the next chapter.



% ================== Why This Matters ==================
\section{Why This Matters for Waves}

In wave theory:
\begin{itemize}
    \item equations involve derivatives,
    \item oscillations repeat endlessly,
    \item phase matters deeply.
\end{itemize}

If our mathematics is clumsy, our understanding suffers.

A better mathematical language will help us see the physics clearly.



% ================== Summary ==================
\section*{What You Should Know Now}

After this chapter, you should:
\begin{itemize}
    \item understand why sine and cosine are inconvenient,
    \item see why phase handling is messy,
    \item feel the need for a better tool.
\end{itemize}

You should be asking:
\begin{quote}
\textbf{Is there a cleaner way?}
\end{quote}



% ================== Looking Ahead ==================
\section*{Looking Ahead}

In the next chapter, we will introduce:
\begin{quote}
\textbf{Complex numbers — not as abstraction, but as a solution.}
\end{quote}

% ================== Title ==================
\pagebreak
\begin{center}
    {\Large \textbf{Chapter 6: Complex Numbers Without Fear}}
\end{center}

\vspace{1.5em}

% ================== Motivation ==================
\section*{Why this chapter exists}

In the previous chapter, we discovered an honest problem:

\begin{quote}
\textbf{Sine and cosine describe oscillations well,  
but they make calculations messy.}
\end{quote}

Physics does not introduce new mathematics for decoration.

It introduces new mathematics only when it makes thinking simpler.

This chapter introduces \textbf{complex numbers} for exactly that reason.



% ================== What Is the Problem ==================
\section{The Problem We Want to Solve}

Recall the oscillation:
\[
x(t) = A \sin(\omega t + \phi)
\]

Differentiating, adding, or shifting this motion requires:
\begin{itemize}
    \item trigonometric identities,
    \item careful bookkeeping of phases,
    \item long algebra.
\end{itemize}

The physics is simple.
The algebra is not.

We want a tool where:
\begin{itemize}
    \item differentiation is easy,
    \item phase shifts are simple,
    \item oscillations keep the same form.
\end{itemize}



% ================== Imaginary Unit ==================
\section{The Imaginary Unit}

We introduce a new symbol:
\[
i
\]

defined by:
\[
i^2 = -1
\]

This does \emph{not} mean something unreal exists.

It is simply an extension of numbers,
just like negative numbers once were.



% ================== Complex Numbers ==================
\section{What Is a Complex Number?}

A complex number has two parts:
\[
z = a + ib
\]

where:
\begin{itemize}
    \item $a$ is the real part,
    \item $b$ is the imaginary part.
\end{itemize}

We write:
\[
\Re(z) = a, \qquad \Im(z) = b
\]

This is just bookkeeping.



% ================== Geometric View ==================
\section{Complex Numbers as Points in a Plane}

Instead of a number line,
complex numbers live in a \textbf{plane}.

\begin{itemize}
    \item horizontal axis: real part,
    \item vertical axis: imaginary part.
\end{itemize}

A complex number is a point or vector in this plane.

This geometric view is crucial.



% ================== Rotation ==================
\section{Multiplication Means Rotation}

Multiplying by $i$ has a beautiful meaning:
\begin{quote}
\textbf{Multiplying by $i$ rotates a vector by $90^\circ$.}
\end{quote}

Multiplying by $i^2$ rotates by $180^\circ$.

This is why complex numbers are perfect for oscillations and waves.



% ================== Exponentials ==================
\section{Exponential Growth and Decay}

Recall the real exponential:
\[
e^t
\]

It has the special property:
\[
\frac{d}{dt} e^t = e^t
\]

This property is exactly what we wanted.

Now we ask a daring question:

\begin{quote}
\textbf{What happens if we allow the exponent to be imaginary?}
\end{quote}



% ================== Euler ==================
\section{Euler’s Formula (The Key Result)}

One of the most important formulas in mathematics is:
\[
e^{i\theta} = \cos\theta + i\sin\theta
\]

This is not magic.
It connects:
\begin{itemize}
    \item exponentials,
    \item trigonometry,
    \item geometry.
\end{itemize}

This formula will appear everywhere in wave theory.



% ================== Meaning ==================
\section{What Euler’s Formula Really Means}

The expression:
\[
e^{i\theta}
\]

represents:
\begin{itemize}
    \item a point on the unit circle,
    \item rotating with angle $\theta$.
\end{itemize}

Rotation replaces oscillation.

That is the key idea.



% ================== Oscillation ==================
\section{Oscillation as Rotation}

Using Euler’s formula:
\[
\sin(\omega t) = \Im\{ e^{i\omega t} \}
\]

So oscillation is:
\begin{quote}
\textbf{the imaginary part of a rotating vector.}
\end{quote}

This turns trigonometry into algebra.



% ================== Why Physics Uses This ==================
\section{Why Physicists Love Complex Exponentials}

With complex exponentials:
\begin{itemize}
    \item differentiation is simple,
    \item phase shifts are multiplication,
    \item equations become cleaner.
\end{itemize}

We do the math using complex numbers,
then take the real part at the end.

Physics stays real.
Math becomes easier.



% ================== Key Rule ==================
\section{The Most Important Rule}

\begin{quote}
\textbf{Complex numbers are a mathematical tool, not physical objects.}
\end{quote}

Only the real part corresponds to measurable motion.

We never lose physical meaning.



% ================== Why This Matters ==================
\section{Why This Matters for Waves}

In wave theory, we will write expressions like:
\[
e^{i(kx - \omega t)}
\]

This compactly represents:
\begin{itemize}
    \item oscillation in time,
    \item repetition in space,
    \item phase relationships.
\end{itemize}

Without complex numbers, this would be painful to write.



% ================== Summary ==================
\section*{What You Should Know Now}

After this chapter, you should:
\begin{itemize}
    \item understand what complex numbers are,
    \item know that $i$ represents rotation,
    \item understand Euler’s formula conceptually,
    \item see oscillation as rotation,
    \item feel no fear of complex exponentials.
\end{itemize}

This chapter removes a major psychological barrier.



% ================== Looking Ahead ==================
\section*{Looking Ahead}

In the next chapter, we will use complex exponentials to describe:

\begin{quote}
\textbf{Waves that move through space.}
\end{quote}
\pagebreak

% ================== Title ==================
\begin{center}
    {\Large \textbf{Chapter 7: Traveling Waves in Space and Time}}
\end{center}

\vspace{1.5em}

% ================== Motivation ==================
\section*{Why this chapter exists}

Until now, we have studied oscillation at a single point.

But waves are not confined to one place.

A wave is:
\begin{quote}
\textbf{oscillation that moves through space.}
\end{quote}

This chapter explains how oscillation and motion combine to form a wave.



% ================== Space Coordinate ==================
\section{Introducing Space}

To describe where something happens, we introduce a space coordinate.

Along a straight line, we use:
\[
x
\]

Now motion can depend on:
\begin{itemize}
    \item time $t$,
    \item position $x$.
\end{itemize}

So displacement becomes:
\[
u(x,t)
\]

This means:
\begin{quote}
``The motion depends on where you are and when you look.''
\end{quote}



% ================== Same Motion, Different Places ==================
\section{The Key Idea of a Traveling Wave}

Imagine many identical oscillators placed along a line.

Each one oscillates:
\begin{itemize}
    \item with the same frequency,
    \item with the same amplitude,
    \item but not at the same time.
\end{itemize}

Each point oscillates with a delay compared to its neighbor.

This delay is what makes the wave move.



% ================== Phase ==================
\section{Phase Depends on Position}

For a single oscillator, the phase is:
\[
\omega t
\]

For a wave, phase depends on both space and time:
\[
\omega t - \text{(something involving } x\text{)}
\]

This tells us:
\begin{itemize}
    \item when the wave reaches a point,
    \item how oscillation shifts along space.
\end{itemize}



% ================== Wavelength ==================
\section{Wavelength}

The \textbf{wavelength} is the distance between repeating points in space.

Examples:
\begin{itemize}
    \item crest to crest,
    \item trough to trough.
\end{itemize}

We denote wavelength by:
\[
\lambda
\]

A small wavelength means rapid variation in space.



% ================== Wavenumber ==================
\section{Wavenumber}

Instead of wavelength, physics often uses:
\[
k = \frac{2\pi}{\lambda}
\]

This is called the \textbf{wavenumber}.

Why use $k$?
\begin{itemize}
    \item it simplifies equations,
    \item it matches angular frequency $\omega$,
    \item it works naturally with exponentials.
\end{itemize}



% ================== Traveling Wave ==================
\section{The Mathematical Form of a Traveling Wave}

A simple traveling wave can be written as:
\[
u(x,t) = A \sin(kx - \omega t)
\]

Let us interpret this carefully:
\begin{itemize}
    \item $A$ is the amplitude,
    \item $kx$ controls variation in space,
    \item $\omega t$ controls variation in time.
\end{itemize}

The minus sign indicates motion in the positive $x$ direction.



% ================== Direction ==================
\section{Direction of Propagation}

If we write:
\[
u(x,t) = A \sin(kx + \omega t)
\]

the wave moves in the opposite direction.

Direction is encoded in the sign.



% ================== Complex Form ==================
\section{Complex Exponential Representation}

Using Euler’s formula, we write:
\[
u(x,t) = \Re\{ A e^{i(kx - \omega t)} \}
\]

This single expression represents:
\begin{itemize}
    \item oscillation,
    \item propagation,
    \item phase relationships.
\end{itemize}

This is the language of wave physics.



% ================== Phase Velocity ==================
\section{Phase Velocity}

The wave pattern moves with speed:
\[
c = \frac{\omega}{k}
\]

This is called the \textbf{phase velocity}.

It tells us how fast a crest moves.



% ================== Important Observation ==================
\section{An Important Observation}

In many physical systems:
\begin{itemize}
    \item $c$ depends on material properties,
    \item $c$ may depend on frequency.
\end{itemize}

This phenomenon is called \textbf{dispersion}.

Rayleigh waves are dispersive.



% ================== Why This Matters ==================
\section{Why This Matters for Rayleigh Waves}

Rayleigh waves are:
\begin{itemize}
    \item traveling waves along the surface,
    \item oscillating in time,
    \item varying with depth.
\end{itemize}

Their mathematical form begins with:
\[
e^{i(kx - \omega t)}
\]

Everything else builds on this.



% ================== Summary ==================
\section*{What You Should Know Now}

After this chapter, you should:
\begin{itemize}
    \item understand what a traveling wave is,
    \item know the meaning of wavelength $\lambda$,
    \item understand wavenumber $k$,
    \item know why $e^{i(kx-\omega t)}$ appears,
    \item understand phase velocity $c$.
\end{itemize}

This chapter is a major milestone.



% ================== Looking Ahead ==================
\section*{Looking Ahead}

In the next chapter, we will study:
\begin{quote}
\textbf{What happens when waves meet boundaries and surfaces.}
\end{quote}
\pagebreak
% ================== Title ==================
\begin{center}
    {\Large \textbf{Chapter 8: Boundaries, Surfaces, and Wave Confinement}}
\end{center}

\vspace{1.5em}

% ================== Motivation ==================
\section*{Why this chapter exists}

In the previous chapter, we learned how waves travel freely through space.

But the real world is not infinite.

It has:
\begin{itemize}
    \item surfaces,
    \item interfaces,
    \item boundaries.
\end{itemize}

This chapter explains how boundaries change wave behavior —  
and why new kinds of waves appear because of them.



% ================== What Is a Boundary ==================
\section{What Is a Boundary?}

A boundary is a place where:
\begin{itemize}
    \item material properties change, or
    \item motion is constrained.
\end{itemize}

Examples:
\begin{itemize}
    \item the surface of the Earth,
    \item the surface of water,
    \item the end of a rope,
    \item a wall.
\end{itemize}

Boundaries impose rules on motion.



% ================== Simple Example ==================
\section{A Simple Example: A Rope Fixed at One End}

If a rope is fixed at one end:
\begin{itemize}
    \item the end cannot move,
    \item displacement must be zero there.
\end{itemize}

This is a boundary condition.

The wave must obey it.



% ================== Reflection ==================
\section{Reflection from a Boundary}

When a wave reaches a boundary:
\begin{itemize}
    \item part of it reflects,
    \item part of it may transmit,
    \item part of it may change form.
\end{itemize}

Reflection is nature enforcing the boundary rule.



% ================== Standing Waves ==================
\section{Standing Waves as Boundary Effects}

When waves reflect back and forth,
they can combine to form \textbf{standing waves}.

Standing waves:
\begin{itemize}
    \item do not travel,
    \item oscillate in place,
    \item exist only because of boundaries.
\end{itemize}

Boundaries create structure.



% ================== Free Surface ==================
\section{What Is a Free Surface?}

A free surface is a boundary where:
\begin{itemize}
    \item the material is not constrained,
    \item no external force acts.
\end{itemize}

Example:
\begin{itemize}
    \item the surface of the Earth exposed to air.
\end{itemize}

The surface cannot pull on the air.

This condition is crucial.



% ================== Physical Meaning ==================
\section{What Does the Free Surface Require?}

At a free surface:
\begin{itemize}
    \item stresses must vanish,
    \item forces must balance to zero.
\end{itemize}

This does not mean displacement is zero.

It means force is zero.

This difference matters greatly.



% ================== Surface Effects ==================
\section{Why Surfaces Create New Waves}

Inside a material:
\begin{itemize}
    \item waves can propagate in all directions.
\end{itemize}

Near a surface:
\begin{itemize}
    \item motion is restricted,
    \item reflections occur continuously,
    \item energy can become trapped.
\end{itemize}

This trapping creates surface waves.



% ================== Energy Trapping ==================
\section{Energy Confinement Near the Surface}

Some wave patterns:
\begin{itemize}
    \item satisfy boundary conditions,
    \item reinforce themselves,
    \item decay away from the surface.
\end{itemize}

These waves:
\begin{itemize}
    \item travel along the surface,
    \item weaken with depth,
    \item carry most energy near the boundary.
\end{itemize}

These are surface waves.



% ================== Rayleigh Wave Preview ==================
\section{A First Look at Rayleigh Waves}

Rayleigh waves:
\begin{itemize}
    \item travel along the Earth's surface,
    \item involve coupled vertical and horizontal motion,
    \item decay exponentially with depth.
\end{itemize}

They exist \emph{only} because of the free surface.

Without the surface, they cannot exist.



% ================== Motion Picture ==================
\section{Particle Motion Near the Surface}

Near the surface:
\begin{itemize}
    \item particles move in elliptical paths,
    \item motion combines vertical and horizontal oscillation.
\end{itemize}

This motion:
\begin{itemize}
    \item feels strong near the ground,
    \item weakens rapidly with depth.
\end{itemize}

This explains why Rayleigh waves are destructive.



% ================== Why Mathematics Is Needed ==================
\section{Why Mathematics Is Needed}

To describe surface waves precisely, we must:
\begin{itemize}
    \item enforce boundary conditions,
    \item combine multiple wave types,
    \item satisfy decay with depth.
\end{itemize}

This requires mathematics.

But the physics comes first.



% ================== Summary ==================
\section*{What You Should Know Now}

After this chapter, you should:
\begin{itemize}
    \item understand what a boundary is,
    \item know why surfaces matter,
    \item understand free-surface conditions,
    \item see why new waves appear near boundaries,
    \item have a physical picture of Rayleigh waves.
\end{itemize}

This chapter explains \emph{why} Rayleigh waves must exist.



% ================== Looking Ahead ==================
\section*{Looking Ahead}

In the next chapter, we will answer a crucial question:

\begin{quote}
\textbf{Why do surface waves decay with depth?}
\end{quote}
\pagebreak
% ================== Title ==================
\begin{center}
    {\Large \textbf{Chapter 9: Evanescent Waves and Decay with Depth}}
\end{center}

\vspace{1.5em}

% ================== Motivation ==================
\section*{Why this chapter exists}

In the previous chapter, we learned that:
\begin{itemize}
    \item surfaces impose constraints,
    \item waves reflect and interfere,
    \item energy can become trapped near boundaries.
\end{itemize}

Now we ask a crucial question:

\begin{quote}
\textbf{Why does the motion weaken as we go deeper into the Earth?}
\end{quote}

The answer leads us to a new kind of wave behavior:
\textbf{evanescent decay}.



% ================== Observation ==================
\section{An Everyday Observation}

During an earthquake:
\begin{itemize}
    \item shaking is strongest at the surface,
    \item shaking decreases with depth.
\end{itemize}

This is not accidental.

It is a fundamental property of surface waves.



% ================== What Decay Means ==================
\section{What Does “Decay” Mean?}

Decay means:
\begin{itemize}
    \item motion becomes smaller,
    \item energy density decreases,
    \item oscillations fade with distance.
\end{itemize}

Importantly:
\begin{quote}
\textbf{Decay does not mean the wave disappears.}
\end{quote}

It means the wave is confined.



% ================== Oscillation vs Decay ==================
\section{Oscillation vs Decay}

There are two very different behaviors:
\begin{itemize}
    \item oscillation: repeating motion,
    \item decay: monotonic decrease.
\end{itemize}

Oscillation uses:
\[
\sin(\cdot), \quad \cos(\cdot)
\]

Decay uses:
\[
e^{-x}
\]

These two behaviors arise from the same equations.



% ================== A Simple Mathematical Example ==================
\section{A Simple Mathematical Example}

Consider the equation:
\[
\frac{d^2u}{dz^2} = \alpha^2 u
\]

Its solutions are:
\[
u(z) = C_1 e^{\alpha z} + C_2 e^{-\alpha z}
\]

This solution does not oscillate.
It grows or decays.



% ================== Physical Meaning ==================
\section{Why Growth Is Forbidden}

In a physical medium:
\begin{itemize}
    \item motion must remain finite,
    \item energy cannot explode.
\end{itemize}

So we reject growing solutions.

Only decaying solutions are allowed.

This is a physical selection rule.



% ================== Evanescent Waves ==================
\section{Evanescent Waves}

A wave that:
\begin{itemize}
    \item oscillates in one direction,
    \item decays in another direction,
\end{itemize}

is called an \textbf{evanescent wave}.

Rayleigh waves are evanescent in depth.



% ================== Vertical Coordinate ==================
\section{Depth as a Coordinate}

We now introduce the vertical coordinate:
\[
z
\]

Convention:
\begin{itemize}
    \item $z = 0$ at the surface,
    \item $z > 0$ downward into the Earth.
\end{itemize}

Decay happens as $z$ increases.



% ================== Vertical Decay ==================
\section{Mathematical Form of Decay}

We write depth-dependent decay as:
\[
e^{-qz}
\]

Here:
\begin{itemize}
    \item $q$ controls how fast decay occurs,
    \item larger $q$ means faster decay.
\end{itemize}

$q$ is called the \textbf{vertical decay rate}.



% ================== Penetration Depth ==================
\section{Penetration Depth}

The distance over which motion remains significant is called the
\textbf{penetration depth}.

It is approximately:
\[
\frac{1}{q}
\]

This tells us:
\begin{itemize}
    \item how deep surface waves penetrate,
    \item how far energy extends below the surface.
\end{itemize}



% ================== Why Decay Is Necessary ==================
\section{Why Decay Is Necessary}

If surface waves did not decay:
\begin{itemize}
    \item they would carry infinite energy,
    \item the Earth would shake everywhere equally.
\end{itemize}

Decay ensures:
\begin{itemize}
    \item energy localization,
    \item physical realism.
\end{itemize}



% ================== Combined Behavior ==================
\section{Oscillation Along the Surface, Decay with Depth}

Rayleigh waves combine:
\begin{itemize}
    \item oscillation along $x$,
    \item decay along $z$.
\end{itemize}

Mathematically, this becomes:
\[
e^{i(kx - \omega t)} e^{-qz}
\]

This is the core structure of surface waves.



% ================== Why This Matters ==================
\section{Why This Matters for Rayleigh Theory}

In Rayleigh wave theory:
\begin{itemize}
    \item $k$ controls surface oscillation,
    \item $q$ controls depth decay,
    \item boundary conditions determine allowed values.
\end{itemize}

Understanding decay is essential before equations appear.



% ================== Summary ==================
\section*{What You Should Know Now}

After this chapter, you should:
\begin{itemize}
    \item understand what evanescent decay means,
    \item know why surface waves decay with depth,
    \item understand the meaning of $e^{-qz}$,
    \item know what penetration depth represents.
\end{itemize}

This removes another major mystery.



% ================== Looking Ahead ==================
\section*{Looking Ahead}

In the next chapter, we will introduce:

\begin{quote}
\textbf{Elastic forces and wave motion in solids.}
\end{quote}
\pagebreak
% ================== Title ==================
\begin{center}
    {\Large \textbf{Chapter 10: Elasticity and Waves in Solids}}
\end{center}

\vspace{1.5em}

% ================== Motivation ==================
\section*{Why this chapter exists}

So far, we have studied:
\begin{itemize}
    \item oscillations,
    \item traveling waves,
    \item boundaries,
    \item decay with depth.
\end{itemize}

But we have not yet answered a fundamental question:

\begin{quote}
\textbf{Why can the Earth support waves at all?}
\end{quote}

The answer is \textbf{elasticity}.



% ================== What Is Elasticity ==================
\section{What Does “Elastic” Mean?}

A material is elastic if:
\begin{itemize}
    \item it resists deformation,
    \item it returns to its original shape when forces are removed.
\end{itemize}

Examples:
\begin{itemize}
    \item a stretched spring,
    \item a compressed rubber band,
    \item solid rock (for small deformations).
\end{itemize}

Elasticity is what allows vibrations and waves to exist.



% ================== Deformation ==================
\section{Deformation and Displacement}

When a solid deforms:
\begin{itemize}
    \item points move slightly from their original positions,
    \item neighboring points move differently.
\end{itemize}

We describe this using the \textbf{displacement field}:
\[
\mathbf{u}(\mathbf{x},t)
\]

This tells us:
\begin{quote}
``How much each point has moved.''
\end{quote}



% ================== Stress and Strain ==================
\section{Strain: Measuring Deformation}

Strain measures:
\begin{itemize}
    \item how much a material stretches,
    \item how much it shears.
\end{itemize}

Strain depends on:
\begin{itemize}
    \item differences in displacement,
    \item not absolute motion.
\end{itemize}

This is why gradients appear later.



% ================== Stress ==================
\section{Stress: The Material’s Response}

Stress measures:
\begin{itemize}
    \item internal forces inside a material,
    \item how strongly the material pushes back.
\end{itemize}

Stress is what transmits forces from one part of the Earth to another.



% ================== Hooke ==================
\section{Hooke’s Law (Idea, Not Formula)}

Elastic materials obey a simple principle:

\begin{quote}
\textbf{Small deformation produces proportional restoring force.}
\end{quote}

This is Hooke’s law.

In three dimensions, this becomes a relation between:
\begin{itemize}
    \item stress,
    \item strain.
\end{itemize}

We will not write the full tensor form yet.



% ================== Two Types of Resistance ==================
\section{Two Fundamental Elastic Responses}

Solids resist deformation in two basic ways:

\subsection*{1. Resistance to Compression}
The material resists being squeezed.

This leads to \textbf{compressional waves}.

\subsection*{2. Resistance to Shear}
The material resists being sheared sideways.

This leads to \textbf{shear waves}.

These two responses create two wave types.



% ================== P Waves ==================
\section{Compressional Waves (P-Waves)}

In compressional waves:
\begin{itemize}
    \item particles move back and forth,
    \item motion is parallel to wave propagation.
\end{itemize}

These waves:
\begin{itemize}
    \item travel fastest,
    \item exist in solids and fluids.
\end{itemize}

They are called \textbf{P-waves}.



% ================== S Waves ==================
\section{Shear Waves (S-Waves)}

In shear waves:
\begin{itemize}
    \item particles move sideways,
    \item motion is perpendicular to propagation.
\end{itemize}

These waves:
\begin{itemize}
    \item travel more slowly than P-waves,
    \item exist only in solids.
\end{itemize}

They are called \textbf{S-waves}.



% ================== Wave Speeds ==================
\section{Wave Speeds and Material Properties}

The speeds of elastic waves depend on:
\begin{itemize}
    \item how stiff the material is,
    \item how dense it is.
\end{itemize}

Stiffer materials transmit waves faster.

Denser materials transmit waves more slowly.



% ================== Elastic Constants ==================
\section{Elastic Constants (Preview)}

To quantify elasticity, we introduce:
\begin{itemize}
    \item elastic moduli,
    \item material parameters.
\end{itemize}

Later, these will appear as:
\[
\lambda, \quad \mu
\]

For now, remember:
\begin{itemize}
    \item $\mu$ measures resistance to shear,
    \item $\lambda$ measures resistance to compression.
\end{itemize}



% ================== Why This Matters ==================
\section{Why This Matters for Rayleigh Waves}

Rayleigh waves:
\begin{itemize}
    \item combine P-wave and S-wave motion,
    \item exist only in elastic solids,
    \item depend on both elastic constants.
\end{itemize}

Understanding elasticity is essential.



% ================== Summary ==================
\section*{What You Should Know Now}

After this chapter, you should:
\begin{itemize}
    \item understand what elasticity means,
    \item know why solids support waves,
    \item understand stress and strain conceptually,
    \item know the difference between P-waves and S-waves,
    \item understand why two wave speeds exist.
\end{itemize}

This chapter completes the physical foundation.
\pagebreak
% ================== Title ==================
\begin{center}
    {\Large \textbf{Chapter 11: Elastic Wave Equations}}
\end{center}

\vspace{1.5em}

% ================== Motivation ==================
\section*{Why this chapter exists}

We now have all the physical ideas in place:
\begin{itemize}
    \item motion and oscillation,
    \item traveling waves,
    \item boundaries and decay,
    \item elasticity, stress, and strain.
\end{itemize}

This chapter answers a single question:

\begin{quote}
\textbf{How do we write all of this as equations?}
\end{quote}

The equations will not introduce new physics.  
They will simply organize what we already know.



% ================== Displacement Field ==================
\section{The Displacement Field}

In an elastic solid, every point can move.

We describe this motion using the \textbf{displacement field}:
\[
\mathbf{u}(\mathbf{x},t)
=
\begin{bmatrix}
u_x(\mathbf{x},t) \\
u_y(\mathbf{x},t) \\
u_z(\mathbf{x},t)
\end{bmatrix}
\]

This tells us:
\begin{quote}
``How much each point has moved from its original position.''
\end{quote}



% ================== Newton ==================
\section{Newton’s Second Law in a Solid}

Newton’s second law says:
\[
\text{force} = \text{mass} \times \text{acceleration}
\]

In a continuous medium:
\begin{itemize}
    \item force becomes \textbf{stress},
    \item mass becomes \textbf{density}.
\end{itemize}

This leads to:
\[
\rho \frac{\partial^2 \mathbf{u}}{\partial t^2}
=
\text{(net internal force per unit volume)}
\]

This is the starting point of elastic wave theory.



% ================== Stress Divergence ==================
\section{Internal Forces and Stress}

Internal forces arise from spatial variations of stress.

Mathematically, this is written as:
\[
\rho \frac{\partial^2 u_i}{\partial t^2}
=
\sum_{j} \frac{\partial \sigma_{ij}}{\partial x_j}
\]

Here:
\begin{itemize}
    \item $\sigma_{ij}$ is the stress tensor,
    \item indices $i,j$ label spatial directions.
\end{itemize}

This equation simply enforces force balance.



% ================== Strain ==================
\section{Strain from Displacement}

Strain measures how displacement varies in space.

For small deformations:
\[
\varepsilon_{ij}
=
\frac{1}{2}
\left(
\frac{\partial u_i}{\partial x_j}
+
\frac{\partial u_j}{\partial x_i}
\right)
\]

This tells us:
\begin{quote}
``How neighboring points move relative to each other.''
\end{quote}



% ================== Hooke ==================
\section{Hooke’s Law in Three Dimensions}

Elasticity connects stress and strain.

For an isotropic solid:
\[
\sigma_{ij}
=
\lambda \delta_{ij} \varepsilon_{kk}
+
2\mu \varepsilon_{ij}
\]

Here:
\begin{itemize}
    \item $\lambda, \mu$ are elastic constants,
    \item $\mu$ is the shear modulus,
    \item $\delta_{ij}$ is the Kronecker delta.
\end{itemize}

This is the mathematical form of Hooke’s law.



% ================== Wave Equation ==================
\section{The Elastic Wave Equation}

Substituting Hooke’s law into Newton’s law gives:
\[
\rho \frac{\partial^2 \mathbf{u}}{\partial t^2}
=
(\lambda + \mu)\nabla(\nabla \cdot \mathbf{u})
+
\mu \nabla^2 \mathbf{u}
\]

This is the \textbf{elastic wave equation}.

It governs all wave motion in an elastic solid.



% ================== Interpretation ==================
\section{Interpreting the Equation}

The equation has two parts:
\begin{itemize}
    \item $\nabla(\nabla \cdot \mathbf{u})$ → compressional motion,
    \item $\nabla^2 \mathbf{u}$ → shear motion.
\end{itemize}

This explains why two wave types exist.



% ================== P and S Waves ==================
\section{P-Waves and S-Waves from the Equation}

From the elastic wave equation, we find:

\subsection*{Compressional Waves (P-waves)}
\[
V_P = \sqrt{\frac{\lambda + 2\mu}{\rho}}
\]

\subsection*{Shear Waves (S-waves)}
\[
V_S = \sqrt{\frac{\mu}{\rho}}
\]

These speeds depend on:
\begin{itemize}
    \item elastic stiffness,
    \item material density.
\end{itemize}



% ================== Why Two Speeds ==================
\section{Why Two Different Wave Speeds Exist}

Solids resist:
\begin{itemize}
    \item compression,
    \item shear.
\end{itemize}

Each resistance produces its own wave.

This is why Rayleigh waves must combine both.



% ================== Time-Harmonic ==================
\section{Time-Harmonic Solutions}

We now assume time-harmonic motion:
\[
\mathbf{u}(\mathbf{x},t)
=
\mathbf{u}(\mathbf{x}) e^{-i\omega t}
\]

This converts time derivatives into algebraic factors:
\[
\frac{\partial^2}{\partial t^2}
\rightarrow
-\omega^2
\]

This step simplifies everything.



% ================== Why This Matters ==================
\section{Why This Matters for Rayleigh Waves}

Rayleigh waves are:
\begin{itemize}
    \item solutions of the elastic wave equation,
    \item constrained by a free surface,
    \item decaying with depth.
\end{itemize}

Everything we need is now in place.



% ================== Summary ==================
\section*{What You Should Know Now}

After this chapter, you should:
\begin{itemize}
    \item understand the elastic wave equation,
    \item know where it comes from physically,
    \item understand the origin of P and S waves,
    \item see why time-harmonic solutions are used.
\end{itemize}

This chapter completes the mathematical foundation.
\pagebreak
% ================== Title ==================
\begin{center}
    {\Large \textbf{Chapter 12: Rayleigh Waves — Putting Everything Together}}
\end{center}

\vspace{1.5em}

% ================== Motivation ==================
\section*{Why this chapter exists}

We have now learned:
\begin{itemize}
    \item oscillations and harmonic motion,
    \item traveling waves in space and time,
    \item boundaries and free surfaces,
    \item exponential decay with depth,
    \item elasticity and elastic wave equations.
\end{itemize}

This chapter answers the final question:

\begin{quote}
\textbf{What happens when all of these ideas act together?}
\end{quote}

The answer is the Rayleigh wave.



% ================== Geometry ==================
\section{The Physical Setup}

We consider:
\begin{itemize}
    \item an elastic solid filling the half-space,
    \item a free surface at $z = 0$,
    \item depth increasing downward ($z > 0$).
\end{itemize}

The wave:
\begin{itemize}
    \item travels along the surface,
    \item decays with depth,
    \item oscillates in time.
\end{itemize}

This geometry is essential.



% ================== Motion ==================
\section{What Kind of Motion Is Allowed?}

Rayleigh waves involve:
\begin{itemize}
    \item horizontal motion,
    \item vertical motion,
    \item no motion perpendicular to the plane.
\end{itemize}

So displacement has two components:
\[
\mathbf{u}(x,z,t)
=
\begin{bmatrix}
u_x(x,z,t) \\
u_z(x,z,t)
\end{bmatrix}
\]

This is called \textbf{plane strain} motion.



% ================== Time Harmonic ==================
\section{Time-Harmonic Assumption}

We assume time-harmonic motion:
\[
\mathbf{u}(x,z,t)
=
\mathbf{u}(x,z)\, e^{-i\omega t}
\]

This removes explicit time dependence
and allows us to focus on spatial structure.

This is a mathematical convenience, not a physical restriction.



% ================== Traveling ==================
\section{Propagation Along the Surface}

Along the surface, the wave travels in the $x$ direction.

So we write:
\[
\mathbf{u}(x,z)
=
\mathbf{U}(z)\, e^{ikx}
\]

Now the full motion is:
\[
\mathbf{u}(x,z,t)
=
\mathbf{U}(z)\, e^{i(kx - \omega t)}
\]

This form combines:
\begin{itemize}
    \item oscillation in time,
    \item propagation in space.
\end{itemize}



% ================== Depth ==================
\section{Decay with Depth}

From earlier chapters, we know:
\begin{itemize}
    \item surface waves must decay with depth,
    \item growing solutions are forbidden.
\end{itemize}

So the depth dependence must involve:
\[
e^{-qz}
\]

Different wave components will have different decay rates.



% ================== P and S Components ==================
\section{Why P and S Waves Both Appear}

The elastic wave equation allows:
\begin{itemize}
    \item compressional (P) motion,
    \item shear (S) motion.
\end{itemize}

Rayleigh waves are not purely P or purely S.

They are a \textbf{coupled combination} of both.

This coupling is required to satisfy the surface condition.



% ================== General Form ==================
\section{General Structure of a Rayleigh Wave}

The displacement is built from:
\begin{itemize}
    \item P-type components decaying with depth,
    \item S-type components decaying with depth.
\end{itemize}

Symbolically:
\[
\mathbf{U}(z)
=
A_P\, \mathbf{U}_P\, e^{-q_P z}
+
A_S\, \mathbf{U}_S\, e^{-q_S z}
\]

Here:
\begin{itemize}
    \item $q_P, q_S$ are decay rates,
    \item $A_P, A_S$ are amplitudes.
\end{itemize}



% ================== Boundary ==================
\section{The Free-Surface Boundary Condition}

At the surface $z = 0$:
\begin{itemize}
    \item no external forces act,
    \item stresses must vanish.
\end{itemize}

This gives two conditions:
\[
\tau_{xz}(0) = 0,
\qquad
\tau_{zz}(0) = 0
\]

These conditions determine which wave speeds are allowed.



% ================== Eigenvalue ==================
\section{The Rayleigh Dispersion Condition}

The boundary conditions lead to:
\begin{quote}
\textbf{a system of linear equations for the amplitudes.}
\end{quote}

A non-trivial solution exists only if:
\[
\det \mathbf{A}(k,\omega) = 0
\]

This is the \textbf{Rayleigh dispersion equation}.

Solving it gives:
\begin{itemize}
    \item allowed wavenumbers $k$,
    \item allowed phase velocities $c = \omega/k$.
\end{itemize}



% ================== Physical Meaning ==================
\section{What the Dispersion Equation Means}

The dispersion equation tells us:
\begin{itemize}
    \item not all wave speeds are allowed,
    \item the surface selects special modes,
    \item each mode has a specific structure.
\end{itemize}

Each solution corresponds to one Rayleigh mode.



% ================== Particle Motion ==================
\section{Particle Motion}

The resulting motion:
\begin{itemize}
    \item is elliptical near the surface,
    \item decays with depth,
    \item reverses sense with depth.
\end{itemize}

This explains:
\begin{itemize}
    \item why Rayleigh waves are destructive,
    \item why they dominate surface shaking.
\end{itemize}



% ================== Big Picture ==================
\section{The Big Picture}

Rayleigh waves exist because:
\begin{itemize}
    \item the Earth is elastic,
    \item a free surface exists,
    \item waves must satisfy boundary conditions,
    \item energy must remain finite.
\end{itemize}

They are not an accident.
They are a necessity.



% ================== Summary ==================
\section*{What You Should Know Now}

After this chapter, you should:
\begin{itemize}
    \item understand what Rayleigh waves are,
    \item know why they travel along surfaces,
    \item understand why they decay with depth,
    \item see how P and S waves combine,
    \item understand what the dispersion equation represents.
\end{itemize}

This chapter completes the foundation.

\pagebreak
% ================== Title ==================
\begin{center}
    {\Large \textbf{Chapter 13: From Physics to Matrices}}
\end{center}

\vspace{1.5em}

% ================== Motivation ==================
\section*{Why this chapter exists}

Up to now, we have:
\begin{itemize}
    \item understood Rayleigh waves physically,
    \item written their qualitative structure,
    \item seen boundary conditions select allowed modes.
\end{itemize}

But modern Rayleigh-wave modeling does not proceed by hand algebra.

It proceeds using:
\begin{itemize}
    \item vectors,
    \item matrices,
    \item linear algebra.
\end{itemize}

This chapter explains \emph{why that is unavoidable}.



% ================== Many Unknowns ==================
\section{Why the Problem Grows Large}

Even in a single homogeneous half-space, a Rayleigh wave contains:
\begin{itemize}
    \item P-wave components,
    \item S-wave components,
    \item upward and downward decaying parts.
\end{itemize}

Each component has an amplitude.

That already gives multiple unknowns.



% ================== Layers ==================
\section{What Changes in Layered Media}

In layered media:
\begin{itemize}
    \item each layer has different material properties,
    \item each layer supports its own P and S decay rates,
    \item waves reflect and transmit at every interface.
\end{itemize}

So for each layer, we get:
\begin{itemize}
    \item new amplitudes,
    \item new decay constants,
    \item new continuity conditions.
\end{itemize}

The number of unknowns grows rapidly.



% ================== Continuity ==================
\section{Interface Conditions}

At every interface between layers:
\begin{itemize}
    \item displacement must be continuous,
    \item stress must be continuous.
\end{itemize}

These are linear conditions.

Each condition relates amplitudes above and below the interface.



% ================== Linear System ==================
\section{Why the Problem Is Linear}

Key observation:

\begin{quote}
\textbf{Rayleigh-wave theory is linear.}
\end{quote}

That means:
\begin{itemize}
    \item displacements add,
    \item stresses add,
    \item boundary conditions are linear.
\end{itemize}

Linear problems naturally lead to linear algebra.



% ================== State Vector ==================
\section{The State Vector Idea}

Instead of tracking every amplitude separately,
we group physical quantities into a vector.

A natural choice is:
\[
\mathbf{U}(z)
=
\begin{bmatrix}
u_x(z) \\
u_z(z) \\
\tau_{xz}(z) \\
\tau_{zz}(z)
\end{bmatrix}
\]

This is called the \textbf{state vector}.

It contains everything the wave is doing at depth $z$.



% ================== Why This Choice ==================
\section{Why These Four Quantities?}

These quantities are chosen because:
\begin{itemize}
    \item they fully describe motion and force,
    \item they are continuous across interfaces,
    \item boundary conditions act directly on them.
\end{itemize}

This choice is not arbitrary.
It is physically optimal.



% ================== Linear Mapping ==================
\section{Layers as Linear Mappings}

Inside a homogeneous layer:
\begin{itemize}
    \item the elastic equations are linear,
    \item depth dependence is exponential.
\end{itemize}

Therefore:
\[
\mathbf{U}(z + h)
=
\mathbf{P}
\mathbf{U}(z)
\]

Here:
\begin{itemize}
    \item $\mathbf{P}$ is a matrix,
    \item it depends on material properties and frequency,
    \item it propagates the state vector through the layer.
\end{itemize}

This is the \textbf{propagator matrix}.



% ================== Stacking ==================
\section{Why Matrices Multiply}

If you have two layers:
\[
\mathbf{U}_{\text{top}}
=
\mathbf{P}_1 \mathbf{P}_2 \mathbf{U}_{\text{bottom}}
\]

Layering becomes matrix multiplication.

This is why layered Earth models are matrix products.



% ================== Boundary ==================
\section{Surface Boundary Conditions in Matrix Form}

At the free surface:
\[
\tau_{xz} = 0,
\qquad
\tau_{zz} = 0
\]

These conditions:
\begin{itemize}
    \item select special combinations of amplitudes,
    \item eliminate trivial solutions.
\end{itemize}

Mathematically, this becomes:
\[
\mathbf{A}(k,\omega)\mathbf{a} = 0
\]



% ================== Eigenvalue ==================
\section{Rayleigh Waves as an Eigenvalue Problem}

A non-trivial solution exists only if:
\[
\det \mathbf{A}(k,\omega) = 0
\]

This is an \textbf{eigenvalue condition}.

The eigenvalue is:
\begin{itemize}
    \item $k$ (or $c = \omega/k$),
\end{itemize}

The eigenvector gives:
\begin{itemize}
    \item relative amplitudes,
    \item mode shape.
\end{itemize}


% ================== Big Shift ==================
\section{The Big Conceptual Shift}

Rayleigh-wave modeling is:
\begin{itemize}
    \item not solving PDEs directly,
    \item but solving matrix eigenvalue problems.
\end{itemize}

This shift is what makes numerical modeling possible.


% ================== Connection to Code ==================
\section{Connection to Computational Modeling}

Your code does exactly this:
\begin{itemize}
    \item builds propagator matrices,
    \item multiplies them across layers,
    \item enforces boundary conditions,
    \item finds roots of $\det \mathbf{A} = 0$.
\end{itemize}

The code is not magic.
It is linear algebra acting on physics.



% ================== Summary ==================
\section*{What You Should Know Now}

After this chapter, you should:
\begin{itemize}
    \item understand why matrices appear,
    \item know what the state vector represents,
    \item understand propagator matrices physically,
    \item see Rayleigh waves as eigenvalue problems.
\end{itemize}
\pagebreak
% ================== Title ==================
\begin{center}
    {\Large \textbf{Chapter 14: The Y-Matrix — Anatomy of a Rayleigh Mode}}
\end{center}

\vspace{1.5em}

% ================== Motivation ==================
\section*{Why this chapter exists}

In the previous chapter, we learned:
\begin{itemize}
    \item Rayleigh waves are eigenmodes,
    \item layered media require matrices,
    \item the state vector contains displacement and stress.
\end{itemize}

This chapter answers a precise question:

\begin{quote}
\textbf{How do wave amplitudes produce displacement and stress?}
\end{quote}

The answer is the \textbf{Y-matrix}.



% ================== State Vector ==================
\section{The State Vector Revisited}

We collect physical quantities into a state vector:
\[
\mathbf{U}(z)
=
\begin{bmatrix}
u_x(z) \\
u_z(z) \\
\tau_{xz}(z) \\
\tau_{zz}(z)
\end{bmatrix}
\]

This vector fully describes the wave at depth $z$.



% ================== Wave Components ==================
\section{Wave Components Inside One Layer}

Inside a homogeneous elastic layer, the Rayleigh solution is built from:
\begin{itemize}
    \item P-type components,
    \item S-type components,
    \item each decaying or growing with depth.
\end{itemize}

We write the general solution as:
\[
\mathbf{u}(x,z,t)
=
\sum_{m}
A_m \, \mathbf{u}_m(z)\, e^{i(kx-\omega t)}
\]

Each $A_m$ is an amplitude.



% ================== Amplitude Vector ==================
\section{The Amplitude Vector}

We group all amplitudes into a vector:
\[
\mathbf{a}
=
\begin{bmatrix}
A_P^- \\
A_P^+ \\
A_S^- \\
A_S^+
\end{bmatrix}
\]

These correspond to:
\begin{itemize}
    \item downward and upward P components,
    \item downward and upward S components.
\end{itemize}



% ================== Separation ==================
\section{Separating Depth Dependence}

Each wave component has exponential depth dependence.

We collect this into a diagonal matrix:
\[
\mathbf{D}(z)
=
\begin{bmatrix}
e^{-q_P z} & 0 & 0 & 0 \\
0 & e^{+q_P z} & 0 & 0 \\
0 & 0 & e^{-q_S z} & 0 \\
0 & 0 & 0 & e^{+q_S z}
\end{bmatrix}
\]

Here:
\[
q_P = \sqrt{k^2 - \frac{\omega^2}{V_P^2}},
\quad
q_S = \sqrt{k^2 - \frac{\omega^2}{V_S^2}}
\]



% ================== Y Matrix ==================
\section{Definition of the Y-Matrix}

The Y-matrix connects amplitudes to physical quantities:
\[
\boxed{
\mathbf{U}(z)
=
\mathbf{Y}
\,
\mathbf{D}(z)
\,
\mathbf{a}
}
\]

This is the central equation of Rayleigh-wave theory.



% ================== Explicit Form ==================
\section{Explicit Form of the Y-Matrix}

For isotropic elasticity:
\[
\mathbf{Y}
=
\begin{bmatrix}
ik & ik & -q_S & q_S \\
-q_P & q_P & -ik & -ik \\
\lambda(-k^2+q_P^2)+2\mu q_P^2 &
\lambda(-k^2+q_P^2)+2\mu q_P^2 &
-2\mu ik q_S &
2\mu ik q_S \\
2\mu ik q_P &
-2\mu ik q_P &
-\mu(q_S^2+k^2) &
-\mu(q_S^2+k^2)
\end{bmatrix}
\]

Every entry has a physical meaning.



% ================== Row Meaning ==================
\section{Meaning of the Rows}

Each row corresponds to a physical quantity:
\begin{itemize}
    \item Row 1: horizontal displacement $u_x$,
    \item Row 2: vertical displacement $u_z$,
    \item Row 3: shear stress $\tau_{xz}$,
    \item Row 4: normal stress $\tau_{zz}$.
\end{itemize}



% ================== Column Meaning ==================
\section{Meaning of the Columns}

Each column corresponds to:
\begin{itemize}
    \item one wave type,
    \item one propagation direction.
\end{itemize}

So:
\begin{itemize}
    \item Column 1: downward P,
    \item Column 2: upward P,
    \item Column 3: downward S,
    \item Column 4: upward S.
\end{itemize}



% ================== Physical Insight ==================
\section{Why the Y-Matrix Is Powerful}

The Y-matrix:
\begin{itemize}
    \item encodes elastodynamics exactly,
    \item separates physics from geometry,
    \item works for any layer.
\end{itemize}

Once built, everything else is linear algebra.



% ================== Interface ==================
\section{Continuity Across Interfaces}

At an interface:
\[
\mathbf{U}_{\text{above}} = \mathbf{U}_{\text{below}}
\]

Because $\mathbf{U}$ is continuous,
the Y-matrix makes interface matching automatic.



% ================== Half Space ==================
\section{Half-Space Condition}

In the half-space:
\begin{itemize}
    \item growing exponentials are forbidden,
    \item only decaying columns remain.
\end{itemize}

This reduces the amplitude vector size.



% ================== Eigenfunctions ==================
\section{From Eigenvectors to Eigenfunctions}

Solving the dispersion equation gives:
\begin{itemize}
    \item eigenvalue: phase velocity,
    \item eigenvector: amplitude ratios.
\end{itemize}

Substituting the eigenvector into:
\[
\mathbf{U}(z)
=
\mathbf{Y}\mathbf{D}(z)\mathbf{a}
\]

gives the full Rayleigh eigenfunction.



% ================== Connection to Code ==================
\section{Connection to Code}

In your code:
\begin{itemize}
    \item the Y-matrix is built explicitly,
    \item $\mathbf{D}(z)$ is diagonal,
    \item matrix products propagate layers,
    \item determinants enforce boundary conditions.
\end{itemize}

The code mirrors this chapter exactly.



% ================== Summary ==================
\section*{What You Should Know Now}

After this chapter, you should:
\begin{itemize}
    \item know what the Y-matrix is,
    \item understand every row and column,
    \item see how amplitudes produce motion,
    \item understand how eigenfunctions are built.
\end{itemize}
\pagebreak
% ================== Title ==================
\begin{center}
    {\Large \textbf{Chapter 15: Propagator Matrices and Layer Stacking}}
\end{center}

\vspace{1.5em}

% ================== Motivation ==================
\section*{Why this chapter exists}

We now understand:
\begin{itemize}
    \item the state vector $\mathbf{U}(z)$,
    \item the Y-matrix,
    \item wave amplitudes and decay.
\end{itemize}

This chapter answers one key question:

\begin{quote}
\textbf{How does the wave evolve from one depth to another?}
\end{quote}

The answer is the \textbf{propagator matrix}.



% ================== Inside a Layer ==================
\section{What Happens Inside One Homogeneous Layer}

Inside a single layer:
\begin{itemize}
    \item material properties are constant,
    \item wave equations do not change with depth,
    \item solutions are exponential.
\end{itemize}

This means the wave evolves in a predictable way.



% ================== State at Top and Bottom ==================
\section{State Vector at Top and Bottom of a Layer}

Let:
\begin{itemize}
    \item $z = 0$ be the top of the layer,
    \item $z = h$ be the bottom of the layer.
\end{itemize}

At the top:
\[
\mathbf{U}(0) = \mathbf{Y}\mathbf{a}
\]

At the bottom:
\[
\mathbf{U}(h) = \mathbf{Y}\mathbf{D}(h)\mathbf{a}
\]

The same amplitudes $\mathbf{a}$ appear in both expressions.



% ================== Eliminating Amplitudes ==================
\section{Eliminating the Amplitudes}

We eliminate $\mathbf{a}$ by solving:
\[
\mathbf{a} = \mathbf{Y}^{-1}\mathbf{U}(0)
\]

Substitute into $\mathbf{U}(h)$:
\[
\mathbf{U}(h)
=
\mathbf{Y}\mathbf{D}(h)\mathbf{Y}^{-1}\mathbf{U}(0)
\]



% ================== Propagator ==================
\section{The Propagator Matrix}

We define the \textbf{propagator matrix}:
\[
\boxed{
\mathbf{P}
=
\mathbf{Y}\mathbf{D}(h)\mathbf{Y}^{-1}
}
\]

So that:
\[
\boxed{
\mathbf{U}(h) = \mathbf{P}\mathbf{U}(0)
}
\]

This is the central object for layered modeling.



% ================== Physical Meaning ==================
\section{Physical Meaning of the Propagator}

The propagator:
\begin{itemize}
    \item transports displacement and stress,
    \item through a layer of thickness $h$,
    \item exactly and linearly.
\end{itemize}

It is not approximate.
It is exact for that layer.



% ================== Direction ==================
\section{Why the Order Matters}

Matrix multiplication is not commutative.

Therefore:
\[
\mathbf{P}_1 \mathbf{P}_2 \neq \mathbf{P}_2 \mathbf{P}_1
\]

This reflects physics:
\begin{itemize}
    \item waves feel layers in order,
    \item the Earth has structure.
\end{itemize}



% ================== Multiple Layers ==================
\section{Stacking Multiple Layers}

For multiple layers:
\[
\mathbf{U}_{\text{top}}
=
\mathbf{P}_1
\mathbf{P}_2
\cdots
\mathbf{P}_N
\mathbf{U}_{\text{bottom}}
\]

We define the global propagator:
\[
\boxed{
\mathbf{M}
=
\prod_{j=N}^{1} \mathbf{P}_j
}
\]

This matrix represents the entire layered Earth.



% ================== Half Space ==================
\section{The Half-Space Condition}

At the bottom:
\begin{itemize}
    \item the medium extends infinitely,
    \item growing exponentials are forbidden.
\end{itemize}

So only decaying P and S components remain.

This reduces the number of amplitudes.



% ================== Surface ==================
\section{Applying the Free-Surface Condition}

At the surface:
\[
\tau_{xz} = 0,
\qquad
\tau_{zz} = 0
\]

This means:
\[
\mathbf{S}\mathbf{U}_{\text{surface}} = \mathbf{0}
\]

where $\mathbf{S}$ selects the stress components.



% ================== Dispersion ==================
\section{The Dispersion Equation}

Combining:
\begin{itemize}
    \item global propagator $\mathbf{M}$,
    \item half-space constraint,
    \item surface boundary condition,
\end{itemize}

leads to:
\[
\det \mathbf{A}(k,\omega) = 0
\]

This equation determines Rayleigh-wave velocities.



% ================== Numerical Stability ==================
\section{Numerical Considerations (Preview)}

Direct propagator multiplication can become unstable:
\begin{itemize}
    \item exponentials can grow large,
    \item matrices may become ill-conditioned.
\end{itemize}

This motivates alternative formulations:
\begin{itemize}
    \item stiffness matrices,
    \item reflection-transmission methods.
\end{itemize}

But the propagator remains the conceptual foundation.



% ================== Connection to Code ==================
\section{Connection to Your Forward Model}

Your code:
\begin{itemize}
    \item builds $\mathbf{Y}$ and $\mathbf{D}$,
    \item forms $\mathbf{P} = \mathbf{Y}\mathbf{D}\mathbf{Y}^{-1}$,
    \item multiplies propagators across layers,
    \item applies surface boundary conditions,
    \item solves $\det = 0$.
\end{itemize}

This chapter is your code, written mathematically.



% ================== Summary ==================
\section*{What You Should Know Now}

After this chapter, you should:
\begin{itemize}
    \item understand propagator matrices physically,
    \item know how layers are stacked,
    \item understand the global propagator,
    \item see how dispersion equations arise.
\end{itemize}

This chapter completes the \textbf{local forward model}.

\end{document}