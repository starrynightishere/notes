\documentclass[11pt,a4paper]{article}

% Packages
\usepackage[utf8]{inputenc}
\usepackage[T1]{fontenc}
\usepackage{lmodern}
\usepackage{microtype}
\usepackage{geometry}
\usepackage{graphicx}
\usepackage{amsmath,amssymb,amsthm,mathtools}
\usepackage{bm}           % bold math
\usepackage{physics}      % \inner, \norm etc.
\usepackage[hidelinks]{hyperref}
\usepackage{bookmark}
\usepackage{cleveref}
\usepackage{enumitem}

% Page layout
\geometry{margin=1in}

% Theorem environments
\theoremstyle{plain}
\newtheorem{theorem}{Theorem}[section]
\newtheorem{lemma}[theorem]{Lemma}
\newtheorem{proposition}[theorem]{Proposition}
\newtheorem{corollary}[theorem]{Corollary}

\theoremstyle{definition}
\newtheorem{definition}[theorem]{Definition}
\newtheorem{example}[theorem]{Example}

\theoremstyle{remark}
\newtheorem{remark}[theorem]{Remark}

% Shortcuts
\newcommand{\R}{\mathbb{R}}
\newcommand{\N}{\mathbb{N}}
\newcommand{\lap}{\Delta}
\newcommand{\inner}[2]{\langle #1, #2\rangle}
\providecommand{\norm}[1]{\left\lVert #1 \right\rVert}
\DeclareMathOperator{\Span}{span}
\DeclareMathOperator{\supp}{supp}

% Title
\title{Eigenfunctions and Eigenvalues:\\ Notes and Examples}
\author{Anurag Mishra}
\date{\today}

\begin{document}
\maketitle
\begin{abstract}
Short notes on eigenfunction problems for differential and integral operators, common examples, variational characterization, and discretizations.
\end{abstract}

\tableofcontents
\bigskip


\section{Computation of Rayleigh Wave Eigenfunctions}

\section{What is an Eigenfunction in This Project?}

In this project, a \textbf{Rayleigh wave eigenfunction} refers to the depth-dependent displacement field
\[
\mathbf{u}(z) =
\begin{bmatrix}
u_x(z) \\
u_z(z)
\end{bmatrix},
\]
which describes how particles in the ground move horizontally and vertically as a Rayleigh wave propagates along the surface.

The eigenfunction is not assumed or prescribed. Instead, it is \emph{derived} from:
\begin{itemize}
  \item the elastodynamic equations,
  \item the layered Earth structure,
  \item traction-free boundary conditions at the surface,
  \item decay conditions in the half-space.
\end{itemize}

The code computes this eigenfunction exactly.

\section{Time-Harmonic Representation}

We assume time-harmonic motion:
\[
\mathbf{u}(x,z,t)
=
\mathbf{u}(z)\,e^{i(kx-\omega t)},
\]
where:
\begin{itemize}
  \item $\omega = 2\pi f$ is angular frequency,
  \item $k = \omega/c$ is the horizontal wavenumber,
  \item $c$ is the Rayleigh phase velocity.
\end{itemize}

This assumption converts the partial differential equations of elasticity into ordinary differential equations in $z$.

\section{Wave Decomposition Inside a Single Layer}

Inside a homogeneous isotropic elastic layer, the general P--SV solution is a superposition of four waves:
\[
\mathbf{u}(z)
=
A_P^- \mathbf{u}_P^- e^{-q_P z}
+
A_P^+ \mathbf{u}_P^+ e^{+q_P z}
+
A_S^- \mathbf{u}_S^- e^{-q_S z}
+
A_S^+ \mathbf{u}_S^+ e^{+q_S z},
\]
where
\[
q_P = \sqrt{k^2 - \frac{\omega^2}{V_p^2}}, \qquad
q_S = \sqrt{k^2 - \frac{\omega^2}{V_s^2}}.
\]

These quantities are computed in the code by:
\begin{verbatim}
qP = vertical_wavenumber(k, omega, Vp)
qS = vertical_wavenumber(k, omega, Vs)
\end{verbatim}

\section{The Y-Matrix: From Wave Amplitudes to Physical Fields}

We collect the wave amplitudes into a vector:
\[
\mathbf{a} =
\begin{bmatrix}
A_P^- \\
A_P^+ \\
A_S^- \\
A_S^+
\end{bmatrix}.
\]

The physical displacement--stress vector is:
\[
\mathbf{U}(z) =
\begin{bmatrix}
u_x(z) \\
u_z(z) \\
\tau_{xz}(z) \\
\tau_{zz}(z)
\end{bmatrix}.
\]

These are related by:
\[
\boxed{
\mathbf{U}(z)
=
\mathbf{Y}
\mathbf{D}(z)
\mathbf{a}
}
\]

The matrix $\mathbf{Y}$ contains material properties and polarization vectors and is implemented in the code as:
\begin{verbatim}
Y, qP, qS = Y_matrix(k, omega, rho, Vp, Vs)
\end{verbatim}

Only the first two rows of $\mathbf{Y}$ are needed to compute displacement.

\section{Exponential Depth Dependence}

The depth dependence is encoded by the diagonal matrix:
\[
\mathbf{D}(z) =
\begin{bmatrix}
e^{-q_P z} & 0 & 0 & 0 \\
0 & e^{+q_P z} & 0 & 0 \\
0 & 0 & e^{-q_S z} & 0 \\
0 & 0 & 0 & e^{+q_S z}
\end{bmatrix}.
\]

In the code, this is implemented as:
\begin{verbatim}
expvec = np.vstack([
    np.exp(-qP * zz),
    np.exp( qP * zz),
    np.exp(-qS * zz),
    np.exp( qS * zz)
])
\end{verbatim}

\section{Surface Boundary Condition and Modal Amplitudes}

At the free surface $z=0$, traction must vanish:
\[
\tau_{xz}(0) = 0, \qquad \tau_{zz}(0) = 0.
\]

This leads to a homogeneous linear system:
\[
\mathbf{A}
\begin{bmatrix}
A_P^- \\
A_S^-
\end{bmatrix}
= \mathbf{0}.
\]

Non-trivial solutions exist only if $\det \mathbf{A} = 0$, which is the Rayleigh dispersion condition.

Once a valid phase velocity $c$ is known, the \emph{eigenvector} is obtained by computing the null space:
\begin{verbatim}
_, _, Vh = np.linalg.svd(A)
a2 = Vh[-1, :]
\end{verbatim}

This vector gives the relative P--SV amplitudes at the surface.

\section{Embedding into the Full Wave Vector}

Only decaying waves are allowed in the half-space, so the full amplitude vector is:
\[
\mathbf{a} =
\begin{bmatrix}
A_P^- \\
0 \\
A_S^- \\
0
\end{bmatrix}.
\]

This is implemented as:
\begin{verbatim}
a = np.array([a2[0], 0.0, a2[1], 0.0])
\end{verbatim}

\section{Layer-by-Layer Construction of the Eigenfunction}

For each layer, the displacement is computed as:
\[
\begin{bmatrix}
u_x(z) \\
u_z(z)
\end{bmatrix}
=
\mathbf{Y}_{1:2}
\mathbf{D}(z)
\mathbf{a}.
\]

In code:
\begin{verbatim}
U = Y[:2, :] @ (a[:, None] * expvec)
ux[mask] = U[0]
uz[mask] = U[1]
\end{verbatim}

This procedure is repeated layer by layer, ensuring continuity and correct decay.

\section{Final Interpretation}

The Rayleigh wave eigenfunction produced by the code:
\begin{itemize}
  \item satisfies the elastodynamic equations,
  \item satisfies free-surface boundary conditions,
  \item decays with depth,
  \item correctly incorporates layered structure.
\end{itemize}

\noindent
It is therefore a true physical eigenfunction, not an assumed shape.

% No bibliography in this document (bibliography intentionally omitted)
\end{document}