\documentclass[12pt,a4paper]{article}

% ================= Packages =================
\usepackage{amsmath,amssymb}
\usepackage{geometry}
\usepackage{setspace}
\usepackage{hyperref}

\geometry{margin=1in}
\onehalfspacing

\begin{document}

% ================= Title =================
\begin{center}
{\Large \textbf{Forward Modeling of Rayleigh Waves}}\\[0.6em]
{\Large \textbf{Chapter 1: What Physical Problem Are We Solving?}}
\end{center}

\vspace{1.5em}

% ============================================================
\section{Why Do We Need a Forward Model at All?}

In wave physics, there are always two distinct problems:

\begin{enumerate}
\item \textbf{Forward problem:}  
Given material properties, predict wave motion.

\item \textbf{Inverse problem:}  
Given observed wave motion, infer material properties.
\end{enumerate}

This project is entirely concerned with the \textbf{forward problem}.

\medskip

\textbf{In words:}

\begin{quote}
If we already know how the Earth is layered, what waves will it produce?
\end{quote}

The answer to this question is essential before attempting inversion.
If the forward model is incorrect, inversion is meaningless.

\medskip

Therefore, the first goal of this project is:

\begin{center}
\textbf{Build a physically correct, numerically stable forward model.}
\end{center}

Everything else comes later.

% ============================================================
\section{What Type of Waves Are We Modeling?}

There are several types of seismic waves:

\begin{itemize}
\item Body waves (P-waves and S-waves)
\item Surface waves (Rayleigh waves and Love waves)
\end{itemize}

This project focuses exclusively on \textbf{Rayleigh waves}.

\subsection*{What is a Rayleigh Wave?}

A Rayleigh wave is a surface-confined elastic wave that:

\begin{itemize}
\item propagates along the Earth's surface,
\item decays exponentially with depth,
\item involves coupled horizontal and vertical motion.
\end{itemize}

Physically, particles move in \emph{retrograde ellipses} near the surface.

\medskip

Rayleigh waves are extremely important because:

\begin{itemize}
\item They dominate near-surface seismic recordings.
\item They are highly sensitive to shear-wave velocity ($V_S$).
\item They form the basis of surface-wave inversion methods.
\end{itemize}

This makes them ideal probes of shallow Earth structure.

% ============================================================
\section{What Kind of Earth Model Are We Assuming?}

We assume a \textbf{layered elastic Earth}.

This assumption means:

\begin{itemize}
\item The Earth is composed of horizontal layers.
\item Each layer is homogeneous.
\item Material properties change only with depth.
\end{itemize}

Mathematically, this is written as:
\[
\rho = \rho(z), \quad
V_P = V_P(z), \quad
V_S = V_S(z)
\]

There is \emph{no dependence} on horizontal position.

\subsection*{Why This Assumption Is Reasonable}

\begin{itemize}
\item Sedimentary layers are approximately horizontal.
\item Many field methods assume lateral homogeneity.
\item This assumption drastically simplifies the mathematics.
\end{itemize}

This is not a limitation — it is a deliberate modeling choice.

% ============================================================
\section{What Is the Fundamental Unknown?}

The fundamental physical unknown is \textbf{displacement}.

At every point in the Earth, particles move slightly when a wave passes.

This motion is described by the displacement vector:
\[
\mathbf{u}(x,z,t)
=
\begin{bmatrix}
u_x(x,z,t) \\
u_z(x,z,t)
\end{bmatrix}
\]

where:

\begin{itemize}
\item $u_x$ is horizontal displacement,
\item $u_z$ is vertical displacement.
\end{itemize}

\medskip

Everything else — stress, strain, velocity, energy —
is derived from $\mathbf{u}$.

Therefore, all code in this project ultimately computes $u_x$ and $u_z$.

% ============================================================
\section{Why Time-Harmonic Motion Is Assumed}

Instead of arbitrary time signals, we assume \textbf{time-harmonic motion}:
\[
\mathbf{u}(x,z,t)
=
\mathbf{u}(z)\,e^{i(kx - \omega t)}
\]

This equation must be understood carefully.

\subsection*{Meaning of Each Symbol}

\begin{itemize}
\item $\omega$ = angular frequency (rad/s)
\item $k$ = horizontal wavenumber (rad/m)
\item $t$ = time
\item $x$ = horizontal coordinate
\item $z$ = depth coordinate
\end{itemize}

\subsection*{Why This Is Not a Restriction}

Any physical wavefield can be written as a sum of harmonic waves using Fourier analysis.

Thus:
\begin{itemize}
\item Solving the harmonic problem solves the general problem.
\item Frequency-domain modeling is mathematically exact.
\end{itemize}

This is why all seismic dispersion theory is formulated in the frequency domain.

% ============================================================
\section{Why Complex Numbers Appear Everywhere}

The exponential $e^{i(kx - \omega t)}$ is complex.

This does \emph{not} mean physical displacement is complex.

\medskip

The physical displacement is:
\[
\Re\{\mathbf{u}(x,z,t)\}
\]

Complex numbers are used because they:

\begin{itemize}
\item turn derivatives into multiplications,
\item simplify algebraic manipulation,
\item make boundary conditions easier to enforce.
\end{itemize}

This is standard practice in wave physics.

% ============================================================
\section{Why the Problem Becomes One-Dimensional}

Because:
\begin{itemize}
\item the Earth varies only with depth,
\item horizontal dependence is sinusoidal,
\end{itemize}

the governing partial differential equations reduce to:

\begin{quote}
\textbf{ordinary differential equations in depth $z$.}
\end{quote}

This is the most important simplification in the entire project.

\medskip

All numerical work is therefore done on:
\[
z \in [0, \infty)
\]

This is why all depth grids in the code are one-dimensional arrays.

% ============================================================
\section{What Is the Output of the Forward Model?}

The forward model produces:

\begin{itemize}
\item Rayleigh-wave phase velocities $c(\omega)$,
\item Rayleigh-wave eigenfunctions $u_x(z)$ and $u_z(z)$,
\item apparent phase velocity curves.
\end{itemize}

Each of these quantities corresponds directly to
plots you have generated using the experiment scripts.

% ============================================================
\section{How This Chapter Connects to the Code}

This chapter explains:

\begin{itemize}
\item Why displacement is the central quantity.
\item Why frequency-domain methods are used.
\item Why depth-only modeling is valid.
\end{itemize}

Every subsequent code file assumes these ideas implicitly.

Without understanding this chapter,
the mathematics and code would appear mysterious.

% ============================================================
\section{End-of-Chapter Summary}

At the end of Chapter 1, we understand:

\begin{itemize}
\item What Rayleigh waves are.
\item Why a layered Earth model is assumed.
\item Why displacement is the fundamental unknown.
\item Why time-harmonic solutions are used.
\item Why the problem reduces to depth-only equations.
\end{itemize}

This chapter establishes the physical foundation.
No numerical methods have been used yet.

\medskip

\begin{center}
\textbf{Everything that follows builds directly on this foundation.}
\end{center}
\pagebreak
% ================= Title =================
\begin{center}
{\Large \textbf{Chapter 2: Elastic Waves and the Origin of P--SV Motion}}
\end{center}

\vspace{1.5em}

% ============================================================
\section{What This Chapter Answers}

In Chapter 1 we asked:

\begin{quote}
\textit{What physical problem are we solving?}
\end{quote}

Now we ask a deeper question:

\begin{quote}
\textbf{What equations describe motion inside an elastic Earth?}
\end{quote}

This chapter explains:
\begin{itemize}
\item where the wave equations come from,
\item why P and S waves appear naturally,
\item why Rayleigh waves are a coupling of P and SV motion,
\item why matrix methods become unavoidable.
\end{itemize}

This chapter explains the \emph{physics behind your Y-matrix}.

% ============================================================
\section{Newton’s Law Applied to a Solid}

The starting point of all wave motion is \textbf{Newton’s second law}:

\[
\text{Force} = \text{Mass} \times \text{Acceleration}
\]

In a solid, force is expressed using \textbf{stress}.
Mass is expressed using \textbf{density}.

Thus, the equation of motion is:
\[
\rho \frac{\partial^2 \mathbf{u}}{\partial t^2}
=
\nabla \cdot \boldsymbol{\sigma}
\]

where:
\begin{itemize}
\item $\rho$ is density,
\item $\mathbf{u}$ is displacement,
\item $\boldsymbol{\sigma}$ is the stress tensor.
\end{itemize}

This equation is exact — no approximation yet.

% ============================================================
\section{What Is Stress?}

Stress measures internal forces inside a solid.

In two dimensions $(x,z)$, stress is written as:
\[
\boldsymbol{\sigma}
=
\begin{bmatrix}
\sigma_{xx} & \sigma_{xz} \\
\sigma_{zx} & \sigma_{zz}
\end{bmatrix}
\]

Important facts:
\begin{itemize}
\item $\sigma_{xz} = \sigma_{zx}$ (symmetry),
\item stress relates to deformation, not displacement directly.
\end{itemize}

Stress is produced when the material is strained.

% ============================================================
\section{From Displacement to Strain}

Strain measures deformation.

For small motions (which seismic waves always are), strain is:
\[
\boldsymbol{\varepsilon}
=
\frac{1}{2}
\left(
\nabla \mathbf{u} + (\nabla \mathbf{u})^T
\right)
\]

Explicitly:
\[
\varepsilon_{xx} = \frac{\partial u_x}{\partial x}, \quad
\varepsilon_{zz} = \frac{\partial u_z}{\partial z}, \quad
\varepsilon_{xz} =
\frac{1}{2}
\left(
\frac{\partial u_x}{\partial z}
+
\frac{\partial u_z}{\partial x}
\right)
\]

These quantities measure stretching and shearing.

% ============================================================
\section{Constitutive Law: Hooke’s Law}

Stress is related to strain by Hooke’s law.

For an isotropic elastic solid:
\[
\boldsymbol{\sigma}
=
\lambda (\nabla \cdot \mathbf{u}) \mathbf{I}
+
2\mu \boldsymbol{\varepsilon}
\]

where:
\begin{itemize}
\item $\lambda$ and $\mu$ are Lamé parameters,
\item $\mu$ is the shear modulus,
\item $\lambda$ controls compressibility.
\end{itemize}

These parameters are related to seismic velocities:
\[
\mu = \rho V_S^2,
\qquad
\lambda = \rho V_P^2 - 2\mu
\]

This is exactly what appears in your code.

% ============================================================
\section{Substituting Into Newton’s Law}

Substituting stress into the equation of motion gives:

\[
\rho \frac{\partial^2 \mathbf{u}}{\partial t^2}
=
(\lambda + \mu)\nabla(\nabla \cdot \mathbf{u})
+
\mu \nabla^2 \mathbf{u}
\]

This is the \textbf{elastic wave equation}.

It is a system of coupled partial differential equations.

% ============================================================
\section{Why P and S Waves Appear}

This equation can be separated into two independent parts:

\begin{itemize}
\item Longitudinal (compressional) motion
\item Transverse (shear) motion
\end{itemize}

These produce:
\begin{itemize}
\item P-waves (speed $V_P$),
\item S-waves (speed $V_S$).
\end{itemize}

This is not an assumption.
It is a mathematical consequence of isotropic elasticity.

% ============================================================
\section{Why Rayleigh Waves Need Both P and S Waves}

Rayleigh waves:
\begin{itemize}
\item propagate along the surface,
\item decay with depth,
\item require boundary conditions at the free surface.
\end{itemize}

Pure P or pure S waves \emph{cannot} satisfy surface conditions alone.

Therefore:
\begin{center}
\textbf{Rayleigh waves are coupled P--SV waves.}
\end{center}

This is why both $V_P$ and $V_S$ appear everywhere in your code.

% ============================================================
\section{Time-Harmonic Reduction of the PDE}

Assume:
\[
\mathbf{u}(x,z,t) = \mathbf{u}(z)e^{i(kx-\omega t)}
\]

Then derivatives become:
\[
\frac{\partial}{\partial t} \rightarrow -i\omega,
\quad
\frac{\partial}{\partial x} \rightarrow ik
\]

The PDE becomes an \textbf{ODE system in depth $z$}.

This is the key mathematical simplification.

% ============================================================
\section{Vertical Wavenumbers Appear}

Solving the reduced equations yields exponential depth dependence:
\[
e^{\pm q_P z}, \quad e^{\pm q_S z}
\]

where:
\[
q_P = \sqrt{k^2 - \frac{\omega^2}{V_P^2}},
\quad
q_S = \sqrt{k^2 - \frac{\omega^2}{V_S^2}}
\]

These are called \textbf{vertical wavenumbers}.

They control how fast motion decays with depth.

This is exactly what your function
\texttt{vertical\_wavenumber()} computes.

% ============================================================
\section{Why Matrix Formulation Is Natural}

Inside one layer, the general solution is:
\[
\mathbf{u}(z)
=
\sum
\text{(P and S waves with exponential depth dependence)}
\]

Rather than writing many equations repeatedly,
we collect everything into matrices.

This leads to the representation:
\[
\mathbf{U}(z)
=
\mathbf{Y}\mathbf{D}(z)\mathbf{a}
\]

This is not a trick.
It is simply linear algebra organizing physics.

% ============================================================
\section{What This Chapter Has Established}

At the end of this chapter, we now understand:

\begin{itemize}
\item Where the elastic wave equations come from
\item Why P and S waves exist
\item Why Rayleigh waves require coupled P--SV motion
\item Why exponential depth dependence appears
\item Why matrices naturally arise
\end{itemize}

We are now ready to:
\begin{center}
\textbf{construct the Y-matrix explicitly.}
\end{center}

\medskip
\pagebreak
% ================= Title =================
\begin{center}
{\Large \textbf{Forward Modeling of Rayleigh Waves}}\\[0.6em]
{\large Extremely Detailed Foundational Notes (Code-Aligned)}\\[1.2em]
{\Large \textbf{Chapter 3: The State Vector and the Y-Matrix}}
\end{center}

\vspace{1.5em}

% ============================================================
\section{What This Chapter Explains}

In Chapter 2 we derived:
\begin{itemize}
\item elastic wave equations,
\item P and S wave decomposition,
\item exponential depth dependence.
\end{itemize}

We now answer a crucial question:

\begin{quote}
\textbf{How do we organize displacement and stress into a form suitable for computation?}
\end{quote}

The answer is the \textbf{state vector} and the \textbf{Y-matrix}.

These two objects are the backbone of your entire codebase.

% ============================================================
\section{Why We Need More Than Displacement}

At a layer boundary, physics requires:

\begin{itemize}
\item displacement continuity,
\item stress continuity.
\end{itemize}

Therefore, tracking only displacement is insufficient.

We must track:
\begin{itemize}
\item displacement components,
\item stress components.
\end{itemize}

This leads directly to the definition of a state vector.

% ============================================================
\section{Definition of the State Vector}

We define the state vector:
\[
\mathbf{U}(z)
=
\begin{bmatrix}
u_x(z) \\
u_z(z) \\
\tau_{xz}(z) \\
\tau_{zz}(z)
\end{bmatrix}
\]

Each component has a precise meaning:

\begin{itemize}
\item $u_x$: horizontal displacement
\item $u_z$: vertical displacement
\item $\tau_{xz}$: shear stress on horizontal planes
\item $\tau_{zz}$: normal stress on horizontal planes
\end{itemize}

This vector contains \textbf{everything} required to enforce physics.

% ============================================================
\section{Why the State Vector Has Exactly Four Components}

In 2D elasticity:
\begin{itemize}
\item displacement has 2 components,
\item stress has 3 independent components,
\end{itemize}

but Rayleigh waves involve only:
\begin{itemize}
\item $u_x$, $u_z$,
\item $\tau_{xz}$, $\tau_{zz}$.
\end{itemize}

The component $\tau_{xx}$ does not appear in surface conditions.

Thus, four components are both:
\begin{itemize}
\item sufficient,
\item minimal.
\end{itemize}

This is why your matrices are $4 \times 4$.

% ============================================================
\section{General Solution Inside One Layer}

Inside a single homogeneous layer, the depth dependence of motion is a superposition of:

\begin{itemize}
\item downward P-wave,
\item upward P-wave,
\item downward S-wave,
\item upward S-wave.
\end{itemize}

We introduce four complex amplitudes:
\[
\mathbf{a}
=
\begin{bmatrix}
A_P^- \\
A_P^+ \\
A_S^- \\
A_S^+
\end{bmatrix}
\]

Each amplitude multiplies an exponential depth factor.

% ============================================================
\section{Depth Dependence Matrix $\mathbf{D}(z)$}

The exponential depth dependence is written compactly as:
\[
\mathbf{D}(z)
=
\begin{bmatrix}
e^{-q_P z} & 0 & 0 & 0 \\
0 & e^{+q_P z} & 0 & 0 \\
0 & 0 & e^{-q_S z} & 0 \\
0 & 0 & 0 & e^{+q_S z}
\end{bmatrix}
\]

This matrix:
\begin{itemize}
\item contains no material constants,
\item only controls depth decay or growth.
\end{itemize}

This separation is intentional and powerful.

% ============================================================
\section{What the Y-Matrix Does}

The Y-matrix converts abstract wave amplitudes into physical quantities.

It satisfies:
\[
\boxed{
\mathbf{U}(z)
=
\mathbf{Y}
\mathbf{D}(z)
\mathbf{a}
}
\]

Interpretation:

\begin{itemize}
\item Columns $\rightarrow$ individual wave types
\item Rows $\rightarrow$ physical quantities
\end{itemize}

The Y-matrix is determined entirely by:
\begin{itemize}
\item elasticity,
\item kinematics,
\item constitutive laws.
\end{itemize}

Nothing in $\mathbf{Y}$ is arbitrary.

% ============================================================
\section{Explicit Structure of the Y-Matrix}

In your code, the Y-matrix is:

\[
\mathbf{Y}
=
\begin{bmatrix}
ik & ik & -q_S & q_S \\
-q_P & q_P & -ik & -ik \\
\lambda(-k^2+q_P^2)+2\mu q_P^2 &
\lambda(-k^2+q_P^2)+2\mu q_P^2 &
-2\mu ik q_S &
2\mu ik q_S \\
2\mu ik q_P &
-2\mu ik q_P &
-\mu(q_S^2+k^2) &
-\mu(q_S^2+k^2)
\end{bmatrix}
\]

where:
\[
\mu = \rho V_S^2,
\qquad
\lambda = \rho V_P^2 - 2\mu
\]

Each entry comes from differentiating displacement
and applying Hooke’s law.

% ============================================================
\section{Interpretation Row by Row}

\subsection*{First Row: $u_x$}

This row maps wave amplitudes to horizontal displacement.

Terms involving $ik$ come from $\partial / \partial x$.

Terms involving $q_S$ arise from vertical derivatives.

\subsection*{Second Row: $u_z$}

This row maps wave amplitudes to vertical displacement.

The signs reflect upward and downward propagation.

\subsection*{Third Row: $\tau_{xz}$}

This row encodes shear stress.

It depends on:
\begin{itemize}
\item shear modulus $\mu$,
\item spatial derivatives of displacement.
\end{itemize}

\subsection*{Fourth Row: $\tau_{zz}$}

This row encodes normal stress.

It involves:
\begin{itemize}
\item both $\lambda$ and $\mu$,
\item compressional and shear contributions.
\end{itemize}

% ============================================================
\section{Interpretation Column by Column}

Each column corresponds to one wave component:

\begin{itemize}
\item Column 1: downward P-wave
\item Column 2: upward P-wave
\item Column 3: downward SV-wave
\item Column 4: upward SV-wave
\end{itemize}

This correspondence is exact and physical.

% ============================================================
\section{Why This Matrix Is Central to the Code}

The Y-matrix appears in:

\begin{itemize}
\item layer propagators,
\item dispersion equation construction,
\item eigenfunction reconstruction,
\item energy computation.
\end{itemize}

If the Y-matrix is correct, the entire model is correct.

% ============================================================
\section{What This Chapter Has Achieved}

At the end of this chapter, we understand:

\begin{itemize}
\item why the state vector has four components,
\item how wave amplitudes are defined,
\item how physical quantities are assembled,
\item why the Y-matrix has its exact structure.
\end{itemize}

This is the mathematical heart of Rayleigh-wave modeling.

\medskip

The next step is to understand how layers communicate.

\begin{center}
\textbf{That requires the propagator matrix.}
\end{center}
\pagebreak
% ================= Title =================
\begin{center}
{\Large \textbf{Chapter 4: Dispersion, Phase Velocity, and Apparent Velocity}}
\end{center}

\vspace{1.5em}

% ============================================================
\section{Why This Chapter Is Extremely Important}

Up to this point, we have:

\begin{itemize}
\item defined the physical problem,
\item derived the governing equations,
\item constructed the Y-matrix,
\item built wave solutions inside layers.
\end{itemize}

But now a student might reasonably ask:

\begin{quote}
\textbf{Why are we doing all this? What do the final plots actually mean?}
\end{quote}

This chapter answers that question.

It explains:
\begin{itemize}
\item what dispersion is,
\item what phase velocity means,
\item why velocity depends on frequency,
\item why we see many curves instead of one,
\item what apparent velocity really represents.
\end{itemize}

This chapter connects the mathematics to physical intuition.

% ============================================================
\section{Start With the Simplest Possible Wave}

Before Rayleigh waves, imagine the simplest wave possible.

Suppose a wave travels a distance $x$ in time $t$.

Then its speed is:
\[
\text{speed} = \frac{x}{t}
\]

If the wave always travels at the same speed, regardless of frequency,
we say the medium is \textbf{non-dispersive}.

Examples:
\begin{itemize}
\item Light in vacuum
\item Sound in air (approximately)
\end{itemize}

In such media, one velocity describes everything.

% ============================================================
\section{What Is Phase Velocity?}

Now consider a sinusoidal wave:
\[
u(x,t) = \sin(kx - \omega t)
\]

A point of constant phase satisfies:
\[
kx - \omega t = \text{constant}
\]

Differentiating:
\[
k \frac{dx}{dt} - \omega = 0
\]

So the speed of that phase point is:
\[
\boxed{
c = \frac{\omega}{k}
}
\]

This is called the \textbf{phase velocity}.

\medskip

\textbf{Important:}

Phase velocity is not the speed of energy.
It is the speed of wave crests.

This is exactly the quantity your code computes.

% ============================================================
\section{What Is Dispersion? (Very Important)}

A medium is said to be \textbf{dispersive} if:

\begin{center}
\textbf{Different frequencies travel at different phase velocities.}
\end{center}

Mathematically:
\[
c = c(\omega)
\]

This means:
\begin{itemize}
\item low-frequency waves travel at one speed,
\item high-frequency waves travel at another.
\end{itemize}

This is called \textbf{dispersion}.

% ============================================================
\section{Why a Layered Earth Is Dispersive}

In a layered Earth:

\begin{itemize}
\item high-frequency waves sample shallow layers,
\item low-frequency waves sample deeper layers.
\end{itemize}

Because deeper layers often have different velocities,
the wave speed changes with frequency.

\medskip

This is the physical origin of Rayleigh-wave dispersion.

\begin{center}
\textbf{Dispersion = depth sensitivity + velocity contrast}
\end{center}

% ============================================================
\section{What We Actually Compute in the Code}

At a fixed frequency $f$:

\begin{itemize}
\item we guess a phase velocity $c$,
\item compute $k = \omega / c$,
\item check if boundary conditions are satisfied.
\end{itemize}

Only certain values of $c$ work.

Each valid value is a \textbf{Rayleigh mode}.

This process is repeated for many frequencies.

% ============================================================
\section{Why We Get Many Modes at One Frequency}

This is crucial.

A layered Earth behaves like a waveguide.

Just like:
\begin{itemize}
\item a guitar string has harmonics,
\item a pipe has overtones,
\end{itemize}

the Earth supports:
\begin{itemize}
\item a fundamental Rayleigh mode,
\item higher-order Rayleigh modes.
\end{itemize}

Each mode:
\begin{itemize}
\item satisfies boundary conditions,
\item has a different depth structure,
\item has a different phase velocity.
\end{itemize}

That is why your plots show \textbf{multiple points at one frequency}.

% ============================================================
\section{Understanding the “All Modes” Plot}

In your ``all modes'' plots:

\begin{itemize}
\item x-axis = frequency (Hz),
\item y-axis = phase velocity (m/s),
\item each dot = one Rayleigh mode.
\end{itemize}

At a given frequency:
\begin{itemize}
\item lowest dot = fundamental mode,
\item higher dots = higher modes.
\end{itemize}

This plot answers:
\begin{quote}
\textit{What Rayleigh waves are mathematically allowed?}
\end{quote}

% ============================================================
\section{But Experiments Do NOT See All These Modes}

Here is a key experimental fact:

\begin{center}
\textbf{Real seismic data usually shows one dominant dispersion curve.}
\end{center}

Why?

Because:
\begin{itemize}
\item sources excite many modes,
\item receivers measure total motion,
\item modes interfere.
\end{itemize}

The instrument does not label modes.

It only records motion.

% ============================================================
\section{What Is Apparent Phase Velocity?}

The \textbf{apparent phase velocity} is:

\begin{quote}
A single effective velocity that represents the combined contribution
of all Rayleigh modes at a given frequency.
\end{quote}

It is not a new physical wave.
It is a summary statistic.

% ============================================================
\section{Why Energy Determines Which Modes Matter}

Not all modes contribute equally.

A mode that:
\begin{itemize}
\item has large displacement near the surface,
\item carries more energy,
\item dominates the recorded signal.
\end{itemize}

Therefore, energy provides a natural weighting.

This leads to the definition:
\[
\boxed{
c_{\text{app}} =
\frac{\sum_m E_m c_m}{\sum_m E_m}
}
\]

This is exactly what your code implements.

% ============================================================
\section{Numerical Example (Conceptual)}

Suppose at a given frequency:

\begin{center}
\begin{tabular}{c c c}
Mode & Velocity (m/s) & Energy \\
\hline
1 & 300 & 0.70 \\
2 & 450 & 0.25 \\
3 & 600 & 0.05
\end{tabular}
\end{center}

Then:
\[
c_{\text{app}} =
\frac{0.70 \times 300 + 0.25 \times 450}{0.95}
\approx 345 \text{ m/s}
\]

Mode 3 contributes very little.

% ============================================================
\section{Why Energy Thresholding Is Used}

Very weak modes:
\begin{itemize}
\item are numerically unstable,
\item are not physically observable.
\end{itemize}

So the code discards modes with:
\[
E < \epsilon E_{\max}
\]

This makes the apparent velocity stable and meaningful.

% ============================================================
\section{Why the Apparent Curve Is Smooth}

The apparent velocity is:
\begin{itemize}
\item an average,
\item dominated by low-order modes,
\item slowly varying with frequency.
\end{itemize}

Therefore, smoothing does not change physics —
it removes numerical noise.

% ============================================================
\section{How This Chapter Connects to Your Plots}

Now your plots should be understood as:

\begin{itemize}
\item Gray dots: all mathematically allowed Rayleigh modes
\item Red curve: what an experiment would actually measure
\end{itemize}

This is why the apparent curve lies inside the cloud of modes.

% ============================================================
\section{End-of-Chapter Summary}

At the end of this chapter, we understand:

\begin{itemize}
\item what dispersion means physically,
\item what phase velocity is,
\item why velocity depends on frequency,
\item why multiple modes exist,
\item what apparent velocity represents,
\item why energy weighting is essential.
\end{itemize}

You now understand what your dispersion plots \emph{mean},
not just how to generate them.

\medskip

\begin{center}
\textbf{The forward model is now physically complete.}
\end{center}
\pagebreak
% ================= Title =================
\begin{center}
{\Large \textbf{Chapter 5: Rayleigh Eigenfunctions and Energy Localization}}
\end{center}

\vspace{1.5em}

% ============================================================
\section{What This Chapter Explains}

In Chapter 4 we learned:

\begin{itemize}
\item what dispersion is,
\item why phase velocity depends on frequency,
\item why multiple Rayleigh modes exist,
\item what apparent velocity represents.
\end{itemize}

Now we ask a deeper physical question:

\begin{quote}
\textbf{What does a Rayleigh wave actually look like inside the Earth?}
\end{quote}

This chapter explains:
\begin{itemize}
\item what eigenfunctions are,
\item how they are computed,
\item why they decay with depth,
\item why some modes dominate observations,
\item why energy controls everything.
\end{itemize}

This chapter explains your **eigenfunction plots**.

% ============================================================
\section{What Is an Eigenfunction? (Very Basic Explanation)}

An \textbf{eigenfunction} is simply:

\begin{quote}
The shape of the wave as a function of depth.
\end{quote}

For Rayleigh waves, this means:
\[
u_x(z) \quad \text{and} \quad u_z(z)
\]

These functions tell us:
\begin{itemize}
\item how far the wave penetrates,
\item where motion is strongest,
\item how particles move at each depth.
\end{itemize}

Without eigenfunctions, dispersion curves are just numbers.

% ============================================================
\section{Why Eigenfunctions Depend on Depth}

Rayleigh waves are surface waves.

This means:
\begin{itemize}
\item they are strongest at the surface,
\item they must decay with depth,
\item they cannot grow infinitely downward.
\end{itemize}

This decay is enforced mathematically by:
\[
e^{-q_P z}, \quad e^{-q_S z}
\]

These exponentials appear explicitly in your code.

% ============================================================
\section{How Eigenfunctions Are Computed in the Code}

Eigenfunctions are computed in the file:
\begin{quote}
\texttt{eigenfunctions/compute\_modeshape.py}
\end{quote}

The computation proceeds in four conceptual steps:

\begin{enumerate}
\item Choose a frequency $f$
\item Choose a valid Rayleigh velocity $c$
\item Determine relative wave amplitudes
\item Reconstruct displacement as a function of depth
\end{enumerate}

Each step corresponds to a physical idea.

% ============================================================
\section{Step 1: Fix Frequency and Phase Velocity}

At this stage:
\begin{itemize}
\item frequency $f$ is fixed,
\item phase velocity $c$ is already known from dispersion.
\end{itemize}

This uniquely determines:
\[
\omega = 2\pi f,
\quad
k = \frac{\omega}{c}
\]

No ambiguity remains.

% ============================================================
\section{Step 2: Enforcing the Free-Surface Condition}

At the Earth's surface:
\[
\tau_{xz}(0) = 0,
\quad
\tau_{zz}(0) = 0
\]

These conditions define a homogeneous linear system:
\[
\mathbf{A}(k,\omega)\mathbf{a} = \mathbf{0}
\]

This system has:
\begin{itemize}
\item infinitely many scaled solutions,
\item but only one \emph{direction} in amplitude space.
\end{itemize}

That direction is the eigenvector.

% ============================================================
\section{Step 3: Computing the Null Vector}

In the code, the null vector is computed using:

\begin{itemize}
\item Singular Value Decomposition (SVD),
\item selecting the smallest singular value direction.
\end{itemize}

This gives:
\[
\mathbf{a}_2 =
\begin{bmatrix}
A_P^- \\
A_S^-
\end{bmatrix}
\]

This vector fixes the relative contribution of P and SV motion.

% ============================================================
\section{Why Only Two Amplitudes Are Needed}

Upward-growing waves are physically forbidden.

Therefore:
\[
A_P^+ = 0,
\quad
A_S^+ = 0
\]

The full amplitude vector is:
\[
\mathbf{a} =
\begin{bmatrix}
A_P^- \\
0 \\
A_S^- \\
0
\end{bmatrix}
\]

This guarantees decay with depth.

% ============================================================
\section{Step 4: Reconstructing the Displacement Field}

Inside each layer:
\[
\mathbf{U}(z) =
\mathbf{Y}
\mathbf{D}(z)
\mathbf{a}
\]

The code extracts:
\[
u_x(z) = U_1(z),
\quad
u_z(z) = U_2(z)
\]

This is done:
\begin{itemize}
\item layer by layer,
\item ensuring continuity,
\item using local depth coordinates.
\end{itemize}

% ============================================================
\section{Why Eigenfunctions Are Complex-Valued}

Eigenfunctions are complex because:
\begin{itemize}
\item time-harmonic motion uses complex exponentials,
\item phase information is preserved.
\end{itemize}

Physical motion is the real part:
\[
\Re\{u_x(z)\}, \quad \Re\{u_z(z)\}
\]

Your plots correctly show only real components.

% ============================================================
\section{Understanding the Eigenfunction Plots}

In your eigenfunction plots:

\begin{itemize}
\item vertical axis = depth,
\item horizontal axis = displacement amplitude,
\item curves = $u_x(z)$ and $u_z(z)$.
\end{itemize}

Typical features:
\begin{itemize}
\item maximum motion near the surface,
\item decay with depth,
\item changes at layer boundaries.
\end{itemize}

All of these are physically correct.

% ============================================================
\section{Why Eigenfunctions Are Not Unique Yet}

If $\mathbf{a}$ is a solution, then:
\[
\alpha \mathbf{a}
\]
is also a solution for any constant $\alpha$.

Thus:
\begin{itemize}
\item eigenfunctions have a shape,
\item but no absolute scale.
\end{itemize}

This is why normalization is required.

% ============================================================
\section{Why Energy Is the Right Normalization}

The physically meaningful quantity is energy.

For Rayleigh waves, the modal kinetic energy is:
\[
E =
\int_0^\infty
\rho(z)
\left(
|u_x(z)|^2 + |u_z(z)|^2
\right) dz
\]

This integral measures:
\begin{itemize}
\item how strong the mode is,
\item how much motion it produces.
\end{itemize}

% ============================================================
\section{Normalization Condition}

The code normalizes eigenfunctions so that:
\[
\int_0^\infty
\rho
\left(
|u_x|^2 + |u_z|^2
\right) dz
=
1
\]

This removes arbitrary scaling while preserving shape.

This is why normalized and unnormalized plots look identical.

% ============================================================
\section{Why Some Modes Dominate}

Modes with:
\begin{itemize}
\item large surface amplitude,
\item shallow penetration,
\item high energy,
\end{itemize}

dominate seismic observations.

Higher modes:
\begin{itemize}
\item penetrate deeper,
\item have smaller surface motion,
\item carry less energy near receivers.
\end{itemize}

This explains why fundamental modes dominate apparent velocity.

% ============================================================
\section{Connection to Apparent Phase Velocity}

The apparent velocity is energy-weighted:
\[
c_{\text{app}} =
\frac{\sum_m E_m c_m}{\sum_m E_m}
\]

Eigenfunctions determine $E_m$.

Therefore:
\begin{quote}
\textbf{Eigenfunctions control which modes matter.}
\end{quote}

% ============================================================
\section{What This Chapter Has Achieved}

At the end of this chapter, we understand:

\begin{itemize}
\item what Rayleigh eigenfunctions are,
\item how they are computed,
\item why they decay with depth,
\item why normalization is necessary,
\item why energy determines dominance.
\end{itemize}

This explains the physical meaning of your eigenfunction plots.

\medskip

\begin{center}
\textbf{Eigenfunctions complete the physical picture of Rayleigh waves.}
\end{center}

% ================= Title =================
\begin{center}
{\Large \textbf{Chapter 6: Validation, Sanity Checks, and Model Readiness}}
\end{center}

\vspace{1.5em}

% ============================================================
\section{Why Validation Is Absolutely Necessary}

A forward model is not considered complete just because it runs.

In scientific modeling, correctness requires that:

\begin{itemize}
\item the equations are physically correct,
\item the numerical implementation is stable,
\item the results agree with known physical behavior.
\end{itemize}

This chapter answers the most important question:

\begin{quote}
\textbf{How do we know that our Rayleigh-wave forward model is correct?}
\end{quote}

% ============================================================
\section{What We Mean by “Sanity Check”}

A sanity check is a test where:

\begin{itemize}
\item the expected outcome is already known,
\item the model is tested against that outcome,
\item agreement builds confidence.
\end{itemize}

Sanity checks are not optional.
They are required for scientific credibility.

% ============================================================
\section{Sanity Check 1: Homogeneous Half-Space}

The simplest possible Earth model is:

\begin{itemize}
\item infinite,
\item homogeneous,
\item isotropic,
\item elastic.
\end{itemize}

In this case, Rayleigh-wave theory gives a well-known result:

\[
c_R \approx 0.92\,V_S
\]

This result is independent of frequency.

% ============================================================
\section{Why This Is a Perfect Test}

This test is ideal because:

\begin{itemize}
\item the result is known analytically,
\item no numerical ambiguity exists,
\item failure immediately indicates an error.
\end{itemize}

If the model cannot reproduce this,
it cannot be trusted for layered media.

% ============================================================
\section{What the Code Produces}

When the homogeneous test is run, the code reports:

\begin{itemize}
\item a constant Rayleigh velocity,
\item very close to $0.92 V_S$,
\item independent of frequency.
\end{itemize}

This confirms:

\begin{itemize}
\item correct Y-matrix construction,
\item correct boundary conditions,
\item correct dispersion equation.
\end{itemize}

This is the strongest possible validation.

% ============================================================
\section{Sanity Check 2: Depth Decay of Eigenfunctions}

Rayleigh waves must decay with depth.

This is not optional.
If eigenfunctions grow downward, the model is unphysical.

\medskip

In all eigenfunction plots, we observe:

\begin{itemize}
\item maximum displacement at the surface,
\item exponential decay with depth,
\item no growth at large depth.
\end{itemize}

This confirms correct handling of vertical wavenumbers.

% ============================================================
\section{Sanity Check 3: Continuity at Layer Boundaries}

At every layer interface, physics requires:

\begin{itemize}
\item displacement continuity,
\item stress continuity.
\end{itemize}

In the eigenfunction plots:

\begin{itemize}
\item $u_x(z)$ is continuous,
\item $u_z(z)$ is continuous,
\item slope changes reflect material contrast.
\end{itemize}

This confirms correct layer propagation.

% ============================================================
\section{Sanity Check 4: Modal Structure}

For layered models, theory predicts:

\begin{itemize}
\item multiple Rayleigh modes,
\item one fundamental mode,
\item higher-order modes.
\end{itemize}

Your all-modes dispersion plots show:

\begin{itemize}
\item distinct branches,
\item smooth frequency dependence,
\item physically reasonable velocity ranges.
\end{itemize}

This confirms correct root-finding and mode identification.

% ============================================================
\section{Sanity Check 5: Energy Localization}

Physically:

\begin{itemize}
\item fundamental modes should carry more surface energy,
\item higher modes should be deeper and weaker.
\end{itemize}

Your normalized eigenfunctions and energy calculations show exactly this.

This validates the energy integral and normalization scheme.

% ============================================================
\section{Sanity Check 6: Apparent Phase Velocity}

Experiments typically report one dispersion curve.

Your apparent velocity computation shows:

\begin{itemize}
\item smooth frequency dependence,
\item dominance of low-order modes,
\item physically realistic trends.
\end{itemize}

The apparent curve lies inside the modal cloud,
exactly as expected.

% ============================================================
\section{What the Model Does Correctly}

At this point, the forward model correctly:

\begin{itemize}
\item solves elastodynamic equations,
\item enforces free-surface conditions,
\item handles layered media,
\item computes all Rayleigh modes,
\item reconstructs eigenfunctions,
\item normalizes modes physically,
\item produces observable dispersion curves.
\end{itemize}

This is a complete local isotropic forward model.

% ============================================================
\section{What the Model Does NOT Yet Include}

It is important to be explicit about limitations.

The current model does not yet include:

\begin{itemize}
\item elastic anisotropy,
\item nonlocal constitutive behavior,
\item intrinsic attenuation,
\item inversion or optimization.
\end{itemize}

These are not flaws.
They are planned extensions.

% ============================================================
\section{Why the Model Is Ready for Extension}

The model is well-suited for extension because:

\begin{itemize}
\item physics is expressed in matrix form,
\item layers are modular,
\item eigenfunctions and energies are available,
\item the forward operator is explicit.
\end{itemize}

This makes it ideal for:

\begin{itemize}
\item anisotropic stiffness tensors,
\item nonlocal kernel integration,
\item adjoint-based inversion,
\item physics-informed neural networks.
\end{itemize}

% ============================================================
\section{Connection to the Research Plan}

The completed forward model directly supports the research goal:

\begin{quote}
\textit{Modeling and inversion of Rayleigh-wave dispersion in layered anisotropic and heterogeneous media, incorporating nonlocal elastic effects and physics-informed neural networks.}
\end{quote}

The present work establishes the local isotropic baseline.

% ============================================================
\section{Final Summary of the Entire Forward Model}

The complete forward pipeline is:

\begin{enumerate}
\item Define layered Earth model
\item Derive elastic wave equations
\item Decompose into P--SV motion
\item Construct Y-matrix
\item Build propagator matrices
\item Stack layers
\item Enforce surface boundary conditions
\item Solve dispersion equation
\item Compute eigenfunctions
\item Normalize by energy
\item Compute apparent phase velocity
\end{enumerate}

Each step is physically justified and numerically validated.

% ============================================================
\section{Final Statement}

We conclude that:

\begin{itemize}
\item the forward Rayleigh-wave model is correct,
\item the numerical implementation is stable,
\item the results agree with theory,
\item the framework is ready for research extensions.
\end{itemize}

\medskip

\begin{center}
\textbf{The forward modeling stage is complete.}
\end{center}

\end{document}